\section{Giới thiệu}
\label{sec:Gioithieu}
\subsection{Lý do chọn đề tài}
\label{subsec:LDCDT}
\begin{frame}{Giới thiệu}
\begin{block}{\ref{subsec:LDCDT}. Lý do chọn đề tài}
\begin{figure}[h]
\centering
\includegraphics[width=0.4\linewidth]{images/topMOOC.png}
\pause
\includegraphics[width=0.4\linewidth]{images/Retry.png}
\end{figure}
\pause
\centering
Đó là lý do tôi chọn đề tài 
\\ \textbf{ “Độ tương tự hành vi của chương trình và thực nghiệm”}
\end{block}
\end{frame}

\subsection{Đối tượng, phạm vi nghiên cứu}
\label{subsec:DTPVNC}
\begin{frame}{Giới thiệu}
\begin{block}{\ref{subsec:DTPVNC}. Đối tượng, phạm vi nghiên cứu }
\begin{itemize}
	\item \textbf{Mục tiêu nghiên cứu chính} là tìm cách đánh giá độ tương
	tự về hành vi giữa hai chương trình máy tính 
	\pause
	\item \textbf{Đối tượng nghiên cứu}
	\begin{itemize}
		\item Kỹ thuật sinh Test Case
		\item Các kỹ thuật đo độ tương tự hành vi
		\item Ứng dụng của các kỹ thuật đo độ tương tự hành vi
	\end{itemize}%
	\pause			
	\item \textbf{Phạm vi nghiên cứu}
	\begin{itemize}
		\item  Đo độ tương tự hành vi dựa vào Test Case
		\item Thực nghiệm, đánh giá trên các chương trình C\#
	\end{itemize}
\end{itemize}
\end{block}
\end{frame}

\subsection{Phương pháp nghiên cứu, thực nghiệm}
\label{subsec:PPNCTN}
\begin{frame}{Giới thiệu}
\begin{block}{\ref{subsec:PPNCTN}. Phương pháp nghiên cứu, thực nghiệm}
\begin{itemize}
	\item \textbf{Phương pháp nghiên cứu}
	\begin{itemize}
		\item Độ tương tự hành vi
		\item Một số kỹ thuật sinh Test Case tự động
		\item Kỹ thuật đo độ tương tự hành vi dựa trên Test Case
		\item So sánh, kết hợp các phép đo độ tương tự hành vi
	\end{itemize}		
	\pause
	\item \textbf{Thực nghiệm}
	\begin{itemize}
		\item Tiến hành cài đặt các kỹ thuật đo độ tương tự hành vi
		\item Thực nghiệm trên dữ liệu thực của CodeHunt, và một số dữ liệu thử khác
		\item Phân tích, đánh giá dựa trên kết quả thực nghiệm
	\end{itemize}		
\end{itemize}
\end{block}
\end{frame}

