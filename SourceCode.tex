Một số mã lệnh quan trọng sử dụng trong đề tài, bao gồm:

\section*{Mã nguồn hàm tạo Project của sinh viên}
\lstinputlisting{MakeProjects.cs}

\section*{Mã nguồn hàm tạo Project của giảng viên}
\lstinputlisting{MakeSecretProjects.cs}

\section*{Mã nguồn hàm build Project của sinh viên}
\lstinputlisting{BuildProjects.cs}

\section*{Mã nguồn hàm build Project của giảng viên}
\lstinputlisting{BuildSecretProjects.cs}

\section*{Mã nguồn hàm tạo Project kết hợp mã nguồn của sinh viên và giảng viên}
\lstinputlisting{MakeMetaProjects.cs}

\section*{Mã nguồn hàm build Project kết hợp mã nguồn của sinh viên và giảng viên}
\lstinputlisting{BuildMetaProjects.cs}

\section*{Mã nguồn hàm thực thi Pex trên Project giảng viên}
\lstinputlisting{BuildMetaProjects.cs}

\section*{Mã nguồn hàm thực thi Pex trên Project kết hợp}
\lstinputlisting{RunPexOnMetaProjects.cs}

\section*{Mã nguồn hàm lưu lại cái giá trị Pex tạo ra}
\lstinputlisting{ExtractPexTests.cs}

\section*{Mã nguồn hàm tính độ do RS}
\lstinputlisting{ComputeMetric1.cs}

\section*{Mã nguồn hàm tính độ do SSE}
\lstinputlisting{ComputeMetric2.cs}

\section*{Mã nguồn hàm tính độ do PSE}
\lstinputlisting{ComputeMetric3.cs}