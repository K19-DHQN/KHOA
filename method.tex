Dựa trên kiến thức về kiểm thử phần mềm, kỹ thuật sinh dữ liệu 
kiểm thử DSE và một số định nghĩa về hành vi của chương trình, độ tương 
tự về hành vi của chương trình. Chương này sẽ tập trung làm rõ các phương 
pháp đo độ tương tự về hành vi giữa hai chương trình. 

\section{Một số phép đo độ tương tự hành vi}
\label{sec:metrics}

Theo Định nghĩa \ref{def:equiv-sim}, để đo độ tương tự về hành vi của
hai chương trình ta cần xác định:
\begin{enumerate}
\item Số phần tử miền vào của hai chương trình
\item Số phần tử trên miền vào (cực đại) mà ở đó hai chương trình thực thi giống nhau.
\end{enumerate}
Dễ thấy, việc thứ $1$ tương đối đơn giản, chỉ cần thực thi hai
chương trình và so sánh kết quả. Khó khăn nằm ở công việc $2$. Với
những chương trình có miền vào nhỏ hoặc hữu hạn thì có thể thực hiện
việc này bằng cách vét cạn tất cả các khả năng. Tuy nhiên, với những
chương trình có miền vào cực lớn hoặc vô hạn thì việc vét cạn như vậy
hầu như bất khả thi. Để giải quyết vấn đề này, chúng ta cần đến kỹ
thuật sinh mẫu thử tự động sao cho đảm bảo bao phủ miền vào nhiều
nhất. Luận văn này áp dụng kỹ thuật thực thi biểu trưng đối với chương
trình (dynamic symbolic execution -- DSE) để giải quyết khó khăn này
(Xem Mục \ref{sec:dse}).

Phần còn lại của mục này trình bày ba phương pháp đo độ tương tự về
hành vi giữa hai chương trình: lấy mẫu ngẫu nhiên, áp dụng DSE trên
đơn chương trình tham chiếu, áp dụng DSE trên chương trình hợp thành
từ chương trình giải pháp (\emph{do người học cung cấp}) và chương
trình tham chiếu (\emph{đưa ra bởi người dạy}).

\subsection{Lấy mẫu ngẫu nhiên}

Phương pháp đơn giản nhất để đo độ tương tự về hành vi giữa hai chương
trình là lấy ngẫu nhiên một số giá trị trên miền vào (\emph{Random
  Sampling -- RS}) và tính số lượng giá trị mà ở đó hai chương trình
thực thi giống nhau, từ đó tính mức độ tương tự. Chúng ta gọi phép đo
độ tương tự theo phương pháp này là \emph{Phép đo RS}, được mô tả hình thức
trong Định nghĩa \ref{def:RS-sim}.

\begin{definition}[Phép đo RS]
  \label{def:RS-sim}
  Cho $P_{1}$ và $P_{2}$ là hai chương trình có cùng miền giá trị đầu
  vào $I$, $I_{s}$ là tập con ngẫu nhiên của $I$, $I_{a}$ là tập con
  $I_{s}$ sao cho $exec(P_{1}, I_a) = exec(P_{2}, I_a)$ và
  $\forall j \in I_{s} \setminus I_{a}$, thì
  $exec(P_{1}, j) \neq exec(P_{2}, j)$. Độ tương tự về hành vi giữa
  hai chương trình theo phép đo RS được định nghĩa là
  $M_{RS}(P_{1}, P_{2}) = \left|I_{a}\right| \diagup
  \left|I_{s}\right| $.
\end{definition}

Từ Định nghĩa \ref{def:RS-sim}, chúng tôi xây dựng thuật toán để tính
độ tương tự theo phép đo RS như mô tả trong Thuật toán \ref{alg:RS},
trong đó $I_s = Random(I)$ là thao tác chọn ngẫu nhiên một số giá trị
vào $I_s$ trong miền $I$. Trong phần thực nghiệm ở Chương
\ref{chap:results}, chúng tôi chọn số lượng giá trị ngẫu nhiên là
$200$, nghĩa là $|I_s| = 200$.

\begin{algorithm}[H]
  \caption{Phép đo RS}
  \label{alg:RS}
  \begin{algorithmic}
  \item $P_{1}, P_{2}:$ Là hai chương trình cần đo độ tương tự
  \item $I$: Miền giá trị đầu vào của $P_{1}, P_{2}$
  \item Set $I_{s} = Random(I)$ \Comment{$I_{s}$: Tập con ngẫu nhiên
      của $I$}
  \item Set $I_{a} = \emptyset$

    \For {$i \in I_s$}
        \If{ ($exec(P_{1}, i) = exec(P_{2}, i)$) } \State
        $I_{a} = I_{a} \cup i$
    \EndIf
    \EndFor
  \item
    $M_{RS}(P_{1}, P_{2}) = \left|I_{a}\right| \diagup
    \left|I_{s}\right| $.
  \end{algorithmic}
\end{algorithm}


Phép đo RS là một một phép đo đơn giản để tính độ tương tự về hành vi 
giữa hai chương trình. Vì không phải phân tích từng câu lệnh của 
chương trình mà chỉ quan tâm đến kiểu của miền giá trị vào để 
chọn một số giá trị ngẫu nhiên nên tốc độ xử lý tương đối nhanh 
và ít tốn tài nguyên hệ thống. Tuy nhiên, nhược điểm của phương 
pháp này là độ bao phủ của dữ liệu sinh ngẫu nhiên không cao. 
Chúng ta phân tích Mã lệnh \ref{RS1} và \ref{RS2} để thấy được 
hạn chế của phép đo RS.

\begin{figure}[h]
	\centering
	\caption{Ví dụ hạn chế của phép đo RS}
	\label{fig:Hanche-RS}
	\begin{minipage}[t]{0.45\linewidth}
		\lstinputlisting[label={RS1},caption = {Chương trình
			$P_{1}$}]{RS1.cs}
	\end{minipage}%
	\hfill\vrule\hfill
	\begin{minipage}[t]{0.45\linewidth}
		\lstinputlisting[label={RS2}, caption = {Chương trình
			$P_{2}$}]{RS2.cs}
	\end{minipage}% 
\end{figure}

Hai chương trình $P_{1}$ và $P_{2}$ có cùng miền giá trị đầu vào 
kiểu \texttt{string}, cùng tính toán và trả về giá trị là một số 
nguyên phụ thuộc vào $ x $. Chúng ta dễ dàng thấy được mã lệnh 
chương trình $P_{1}$ khác với chương trình $P_{2}$ khi có một 
điểm rẻ nhánh \texttt{if (x == "XYZ")}. Ở điểm rẻ nhánh này khả 
năng giá trị của \texttt{x = "XYZ"} được lấy ngẫu nhiên trong miền 
đầu vào kiểu \texttt{string} là rất thấp. Do đó chương trình $P_{1}$ 
có thể sẽ không thực thi nhánh \texttt{if (x == "XYZ")}, dữ liệu vào 
được lấy ngẫu nhiên trên miền vào của hai chương trình là không cao.

\subsection{DSE trên chương trình tham chiếu}

Nhược điểm của phép đo RS được khắc phục bằng cách áp dụng kỹ thuật
DSE để tăng độ bao phủ của tập dữ liệu thử. Phép đo áp dụng DSE trên
chương trình tham chiếu, sau đây gọi tắt là SSE (\emph{Single program
  Symbolic Execution}), chỉ cần dựa vào chương trình tham chiếu để xác
định tập dữ liệu thử. Sự khác nhau giữa SSE và RS là SSE cần phần tích
các câu lệnh của chương trình trong khi đó RS chỉ quan tâm đến kiểu dữ
liệu vào. Mô tả hình thức cho phép đo SSE được nêu ra trong Định nghĩa
\ref{def:sse}.

\begin{definition}[Phép đo SSE]
  \label{def:sse}
  Cho $P_{1}$ và $P_{2}$ là hai chương trình có cùng miền giá trị đầu
  vào $I$, trong đó $P_{1}$ là chương trình tham chiếu. Gọi $I_{s}$ là
  tập các giá trị đầu vào được tạo bởi DSE trên $P_{1}$, và
  $I_{a} \subseteq I_s$ là tập con lớn nhất của $I_{s}$ sao cho
  $exec(P_{1}, I_a) = exec(P_{2}, I_a)$ và
  $\forall j \in I_{s} \setminus I_{a}, exec(P_{1}, j) \neq
  exec(P_{2}, j)$. Độ tương tự về hành vi giữa hai chương trình dùng
  phép đo SSE được định nghĩa là
  $M_{SSE}(P_{1}, P_{2}) = \left|I_{a}\right| \diagup
  \left|I_{s}\right| $.
\end{definition}

Thuật toán để đo độ tương tự dùng phép đo SSE được mô tả trong
Thuật toán \ref{alg:sse}, trong đó $P_1$ là chương trình tham chiếu và
$DSE(P_{1})$ là hành động sinh dữ liệu thử cho chương trình $P_1$ áp
dụng kỹ thuật DSE.

\begin{algorithm}[h]
  \caption{Phép đo SSE}
  \label{alg:sse}
  \begin{algorithmic}
  \item $P_{1}, P_{2}:$ hai chương trình cần đo tương tự, $P_1$ là chương trình tham chiếu
  \item $I$: Miền giá trị đầu vào của $P_{1}, P_{2}$
  \item Set $I_{s} = DSE(P_{1})$ \Comment{$I_{s}$: Tập đầu vào của
      $P_{1}$ theo DSE}
  \item Set $I_{a} = \emptyset$
    \For{ ($i \in I_{s}$) }
    \If{ ($exec(P_{1}, i) = exec(P_{2}, i)$) } \State
    $I_{a} = I_{a} \cup i$
    \EndIf
    \EndFor
  \item
    $M_{SSE}(P_{1}, P_{2}) = \left|I_{a}\right| \diagup
    \left|I_{s}\right| $.
  \end{algorithmic}
\end{algorithm}

Để tính độ tương tự hành vi của hai chương trình với phép đo
SSE, chúng ta chọn chương trình mẫu làm chương trình tham
chiếu và áp dụng kỹ thuật DSE để tạo ra các đầu vào thử
nghiệm dựa trên chương trình tham chiếu. Sau đó thực thi cả hai chương
trình dựa trên các giá trị đầu vào thử nghiệm. Tỷ lệ số lượng các kết
quả đầu ra giống nhau của cả hai chương trình trên tổng số các giá trị
đầu vào thử nghiệm của chương trình tham chiếu là kết quả của phép đo
SSE.

Ngược lại với phép đo RS, phép đo SSE khám phá những đường đi khả thi
khác nhau trong chương trình tham chiếu để tạo dữ liệu đầu vào của
chương trình. Do đó, các đầu vào thử nghiệm này sẽ thực thi hết các
đường đi của chương trình tham chiếu và có khả năng phát hiện được
những chương trình cần tính có những hành vi khác so với chương trình
tham chiếu. Nhưng phép đo SSE vẫn còn hạn chế, đó là phép đo SSE không
xem xét đường đi của chương trình cần phân tích để tạo các giá trị đầu
vào thử nghiệm mà chỉ dựa vào các đầu vào thử nghiệm được phân tích từ
chương trình tham chiếu. Các đầu vào thử nghiệm này không bao phủ được
hết các hành vi của chương trình cần phân tích vì chương trình cần
phân tích có thể sẽ có những hành vi khác so với chương trình tham
chiếu. Một số chương trình có thể có những vòng lập vô hạn phụ thuộc
vào giá trị đầu vào nên SSE không thể liệt kê được tất cả các đường
đi của chương trình. Chúng ta phân tích Mã lệnh
\ref{SSE1} và \ref{SSE2} để thấy được hạn chế của phép đo SSE.
\begin{figure}[h]
	\centering
	\caption{Ví dụ hạn chế của phép đo SSE}
	\label{fig:hanche-SSE}
\begin{minipage}[t]{0.45\linewidth}
  \lstinputlisting[label={SSE1}, caption = {Chương trình
    $P_{1}$}]{SSE1.cs}
\end{minipage}%
\hfill\vrule\hfill
\begin{minipage}[t]{0.45\linewidth}
	\lstinputlisting[label={SSE2}, caption = {Chương trình $P_{2}$}]{SSE2.cs}
\end{minipage}%
\end{figure}

Mã lệnh \ref{SSE1} và \ref{SSE2} có cùng tham số đầu vào $ x $ kiểu 
\texttt{int}, cùng tính toán và trả về giá trị $ y $ phụ thuộc vào $ x $, 
chọn chương trình $P_{1}$ làm chương trình tham chiếu, sử dụng kỹ 
thuật DSE để phân tích chương trình $P_{1}$ ta có tập giá trị đầu vào 
là $(0, 1)$ (xem Mục \ref{sec:dse}). Thực thi chương trình $P_{1}$ và 
$P_{2}$ dựa trên các giá trị đầu vào này, ta có tất cả các kết quả 
đầu ra của hai chương trình là giống nhau. Độ tương tự hành vi 
giữa hai chương trình $P_{1}$ và $P_{2}$ sử dụng phép đo SSE trong 
trường hợp này sẽ bằng $ 1 $.

Tuy nhiên, tại điểm nhánh $ (x == 2) $ của chương trình $ P_{2} $ 
đã không được thực thi. Vì vậy, tập các giá trị đầu vào thử nghiệm được tạo 
ra bởi phép đo SSE dựa vào chương trình tham chiếu có thể không 
thực thi hết các đường đi của chương trình cần phân tích.

\subsection{DSE trên chương trình hợp thành}

Để giải quyết giới hạn của phép đo SSE khi tạo ra tập các giá trị đầu
vào thử nghiệm không thực thi hết các các đi của chương trình cần phân
tích, ta xây dựng một chương trình hợp thành kết hợp cả hai, được mô
tả trong Định nghĩa \ref{def:combination}.

\begin{definition}[Chương trình hợp thành]
  \label{def:combination}
  Cho $P_1$ và $P_2$ là hai chương trình có cùng miền vào $I$, trong đó
  $P_1$ là chương trình tham chiếu. Hợp thành của hai chương trình là
  một chương trình mới $P_c = P_1 \oplus P_2$ có dạng
  $assert(exec(P_{1}, I) = exec(P_{2}, I))$, trong đó $assert(\dots)$
  là hàm đánh giá nhận vào một biểu thức điều kiện và đánh giá xem
  biểu thức đó đúng hay sai. Ký hiệu $exec(P_c,i) = \top$ để chỉ
  chương trình hợp thành $P_c$ thỏa hàm đánh giá, cũng có nghĩa là hai
  chương trình thành phần thực thi giống nhau trên đầu vào $i$.
\end{definition}

Phép đo độ tương tự về hành vi giữa hai chương trình áp dụng kỹ thuật
DSE trên chương trình hợp thành, từ đây gọi là phép đo PSE
(\emph{Paired program Symbolic Execution}), giải quyết giới hạn của
phép đo SSE bằng cách tạo một chương trình hợp thành giữa chương trình
cần phân tích với chương trình tham chiếu. Dựa trên chương trình hợp
thành, ta sử dụng kỹ thuật DSE để tạo ra đầu vào thử nghiệm cho cả hai
chương trình. Các đầu vào thử nghiệm này bao gồm các đầu vào thử
nghiệm đúng và không đúng. Các đầu vào thử nghiệm đúng là những giá
trị khi thực thi trên cả hai chương trình sẽ cho kết quả đầu ra như
nhau, ngược lại các đầu vào thử nghiệm không đúng là những giá trị khi
thực thi trên cả hai chương trình sẽ cho kết quả khác nhau. Do đó,
phép đo PSE được tính bằng tỷ lệ các giá trị đầu vào thử
nghiệm đúng trên tổng số các giá trị đầu vào được thử nghiệm.

Thuật toán mô tả cách đo độ tương tự về hành vi giữa hai chương trình 
theo phép đo PSE được mô tả trong Thuật toán \ref{alg:pse}.

 \begin{algorithm}[h]
   \caption{Phép đo PSE}
   \label{alg:pse}
   \begin{algorithmic}
   \item $P_{1}, P_{2}$: hai chương trình cần đo tương tự, $P_1$ là
     chương trình tham chiếu
   \item $I$: Miền giá trị đầu vào của $P_{1}, P_{2}$
   \item $P_{3} = P_1 \oplus P_2$
   \item Set $I_{s} = DSE(P_{3})$ \Comment{$I_{s}$: Tập đầu vào của
       $P_{3}$ theo DSE}
   \item Set $I_{a} = \emptyset$ 
   		\For{$i \in I_{s}$}
     	\If{($exec(P_{1}, i) = exec(P_{2}, i)$) } 
     	\State $I_{a} = I_{a} \cup i$ 
     	\EndIf 
     	\EndFor
   \item
     $M_{PSE}(P_{1}, P_{2}) = \left|I_{a}\right| \diagup
     \left|I_{s}\right| $.
   \end{algorithmic}
 \end{algorithm}

Phép đo PSE đã cải thiện được hạn chế của phép đo
 SSE, khi tập các giá trị đầu vào thử nghiệm được tạo ra dựa trên 
 chương trình hợp thành có thể thực thi hết các đường đi của chương 
 trình tham chiếu và chương trình cần phân tích. Để hiểu rõ hơn phép 
đo PSE chúng ta phân tích một số mã lệnh trong trong Hình \ref{fig:PSE}, 
trong đó Mã lệnh \ref{lst:PSE} là mã lệnh chương trình hợp thành, 
Mã lệnh \ref{lst:PSE-T1} là chương trình tham chiếu và mã lệnh 
\ref{lst:PSE-T2} là chương trình cần tính.

 \begin{figure}[h]
	\centering
	\caption{Ví dụ DSE trên chương trình hợp thành}
	\label{fig:PSE}
	\begin{minipage}[t]{0.3\linewidth}
		\lstinputlisting[caption={Chương trình hợp thành}, label={lst:PSE}]{PSE3.cs}
	\end{minipage}%
	\hfill\vrule\hfill
	\begin{minipage}[t]{0.3\linewidth}
		\lstinputlisting[caption={Chương trình tham chiếu $ T1 $}, label={lst:PSE-T1}]{PSE1.cs}
	\end{minipage}%
	\hfill\vrule\hfill
	\begin{minipage}[t]{0.3\linewidth}
		\lstinputlisting[caption={Chương trình cần tính $ T2 $}, label={lst:PSE-T2}]{PSE2.cs}
	\end{minipage}%
\end{figure}

Trong Hình \ref{fig:PSE}, hàm $ T1 $ và $ T2 $ có cùng tham số 
đầu vào $ x $ kiểu \texttt{int}, cùng tính toán và trả về giá trị 
$ y $ phụ thuộc vào $ x $. Nếu $ x = 1$ thì hàm $ T1 $ trả về 
\texttt{"T"}, nếu $ x = 2$ thì hàm $ T2 $ trả về \texttt{"T"}, 
còn không trả về giá trị \texttt{"F"}. Dựa trên chương trình hợp 
thành, sử dụng kỹ thuật DSE để tạo ra các giá trị đầu vào thử nghiệm 
cho cả hai chương trình, kết quả thu được là các giá trị $ 0, 1, 2 $. 
Thực thi chương trình $ T1 $ và $ T2 $ với các giá trị đầu vào này chỉ 
có giá trị $ 0 $ hai chương trình thực thi cho kết quả như nhau. Độ tương 
tự hành vi giữa hai chương trình $ T1 $ và $ T2 $ dùng phép đo PSE sẽ là $ 0.33 $
 

 Tuy nhiên, phép đo PSE cũng có hạn chế trong quá trình xử lý các 
 vòng lặp lớn hoặc vô hạn. Để giảm bớt hạn chế này, chúng ta có thể 
 giới hạn miền đầu vào hoặc đếm số vòng lặp của các chương trình 
 cần thực thi. Ngoài ra, phép đo PSE khám phá đường dẫn của chương 
 trình hợp thành nên quá trình xử lý sẽ tốn thời gian và tài nguyên 
 của hệ thống hơn so với phép đo SSE khi chỉ khám phá đường dẫn của 
 chương trình tham chiếu.

 \section*{Tổng kết chương}
Chương này tập trung trình bày $3$ phép đo độ tương tự về hành vi 
của hai chương trình: RS, SSE và PSE. Trong đó, phép đo RS là phép đo 
đơn giản nhất để đo độ tương tự về hành vi giữa hai chương trình, bằng
cách lấy ngẫu nhiên một số giá trị trên miền vào. Phép đo RS có ưu điểm 
thực thi chương trình nhanh và đánh giá khách quan, nhưng hạn chế của 
phép đo RS đó là độ bao phủ của dữ liệu sinh ngẫu nhiên không cao. 
Để khắc phục hạn chế của phép đo RS, phép đo SSE sử dụng kỹ thuật 
DSE trên chương trình tham chiếu để tạo dữ liệu đầu vào có độ phủ
cao. Tuy nhiên, dữ liệu vào của phép đo SSE có thể 
không phủ được tất cả các nhánh trong chương trình cần tính. 
Vì vậy, phép đo PSE tạo ra chương trình hợp thành kết hợp giữa 
chương trình tham chiếu và chương trình cần phân tích, sử dụng kỹ 
thuật DSE trên chương trình hợp thành để tạo dữ liệu đầu vào thử 
nghiệm có thể phủ tất cả các nhánh đường đi trên cả hai chương trình, 
từ đó tính mức độ tương tự giữa hai chương trình. 




  
%%% Local Variables:
%%% mode: latex
%%% TeX-master: "Main"
%%% End:
