% Giới thiệu về nội dung sẽ trình bày trong chương này.

Chương này trình bày \dots

\todo{Viết lại chương này cho cẩn thận}

\section{Lý do chọn đề tài}

\subsection{Giới thiệu chung}

Hiện nay, ngành Công nghệ thông tin với các chương trình đào tạo lập trình Online, kỹ sư phần mềm... đang trở nên rất phổ biến. Trên thế giới cũng có nhiều chương trình đào tạo nổi tiếng như Massive Open Online Courses (MOOC), edX, Coursera, Udacity \todo{Thêm links, bib} thu hút hằng ngàn sinh viên theo học. Một số chương trình học lập trình online như Pex4Fun hay Code Hunt là một nền tảng học lập trình online thông qua trò chơi.

Những lớp học như thế này thường đặt ra một số thách thức, làm thế nào để kiểm soát được chất lượng học tập của người học. Trong khi các lớp học có hằng trăm người tham gia, nhưng thành viên tham giảng dạy chỉ vài người trong một lớp học. Hằng ngày, người giáo viên phải thường xuyên kiểm tra, nhắc nhở và phải đọc, nghiên cứu giúp đỡ cho học viên. Chỉ riêng việc đọc và hiểu code do học viên viết ra đã tốn quá nhiều thời gian của người dạy, nếu như bỏ qua thì khó có thể đánh giá được chất lượng của người học.\\ \todo{Bỏ tất cả dấu xuống dòng}

Để giảm bớt những vất vả, khó khăn của người dạy và người học. Một công cụ hỗ trợ, tự động đánh giá hành vi chương trình của người học sẽ giúp tiết kiệm được thời gian, giúp cho giáo viên việc quản lý chất lượng học tập của học viên được tốt hơn. Ví dụ, công cụ sẽ tự động đánh giá hành vi, so sánh hành vi chương trình của người học viết với hành vi chương trình của giáo viên. Nếu hành vi của hai chương trình có tỷ lệ giống nhau càng cao thì điểm số cho chương trình của người học càng cao. Ngược lại, nếu điểm số thấp người dạy và học sẽ tìm hiểu nguyên nhân vì sao và có hướng khác phục những hạn chế mà người học đang gặp phải. Đây cũng là một cách giúp cho người học không đi lệch khỏi kiến thức của chương trình đào tạo, tiết kiệm được thời gian cả người dạy và người học.\\

Ý tưởng trong việc đánh giá độ tương tự hành vi của hai chương trình đó là tính toán tìm ra một miền giá trị đầu vào chung cho cả hai chương trình. Đưa từng giá trị đầu vào chạy đồng thời trên cả hai chương trình và so sánh kết quả đầu ra của hai chương trình. Tuy nhiên, việc này sẽ không đạt, không khả thi, vì trong những trường hợp chương trình có miền giá trị đầu vào lớn hoặc vô hạn. Vì vậy, để đánh giá độ tương tự hành vi của hai chương trình khả thi hơn nếu chúng ta chỉ sử dụng một sô giá trị đầu vào đại diện cho toàn bộ miền giá trị đầu vào chung của hai chương trình.\\

Kỹ thuật Dynamic Symbolic Execution (DSE) là một kỹ thuật thu thập các ràng buộc từ các nhánh của chương trình, phủ nhận lại có hệ thống một phần các ràng buộc để tạo ra giá trị đầu vào cho một chương trình. Hiện nay, đã có nhiều công cụ sử dụng kỹ thuật DSE để giải quyết các ràng buộc một cách mạnh mẽ, tạo ra các giá trị đầu vào tin cậy có độ phủ cao.

\begin{center}
\begin{tabular}  {|c|c|c|} 
	\hline 
	\textbf{Tên Công cụ} & \textbf{Ngôn ngữ} & \textbf{Url} \\ 
	\hline 
	KLEE & LLVM & klee.github.io/ \\ 
	\hline 
	JPF	 & Java	& babelfish.arc.nasa.gov/trac/jpf \\
	\hline 
	jCUTE &	Java &	github.com/osl/jcute \\
	\hline 
	janala2	 & Java &	github.com/ksen007/janala2 \\
	\hline 
	JBSE	& Java	 & github.com/pietrobraione/jbse \\
	\hline 
	KeY &	Java &	www.key-project.org/ \\	
	\hline 
	Mayhem & 	Binary &	forallsecure.com/mayhem.html \\
	\hline 
	Otter &	C	& bitbucket.org/khooyp/otter/overview \\
	\hline 
	Rubyx & 	Ruby &	www.cs.umd.edu/~avik/papers/ssarorwa.pdf \\
	\hline 
	Pex	& .NET Framework	 & research.microsoft.com/en-us/projects/pex/ \\
	\hline 
	Jalangi2 &	JavaScript &	github.com/Samsung/jalangi2 \\
	\hline 
	Kite &	LLVM &	www.cs.ubc.ca/labs/isd/Projects/Kite/ \\
	\hline 
	pysymemu &	x86-64 / Native	 &github.com/feliam/pysymemu/ \\
	\hline 
	Triton	& x86 and x86-64 &	triton.quarkslab.com \\	
	\hline 
	BE-PUM &	x86	 & https://github.com/NMHai/BE-PUM	 \\	
	\hline

\end{tabular} 
\end{center}

Trong những năm gần đây, xu hướng đào tạo lập trình viên nói riêng và công nghệ phần mềm nói chung đang ngày càng trở nên phổ biến. Các trường đại học và một số trung tâm đào tạo đã cho ra đời nhiều chương trình đào tạo phong phú về nội dung, đa dạng về hình thức và thu hút được nhiều sự quan tâm.\\

Trong mỗi khóa học, số lượng học viên thường có hằng trăm, hằng ngàn người tham gia, nhưng chỉ một vài giáo viên giảng dạy. Trong đó, chất lượng việc quản lý, truyền đạt nội dung của giáo viên và mức độ hiểu biết, nắm bắt nội dung chương trình học của học viên là yêu cầu tất cả các khóa học cần đạt được. Để đạt được điều đó, một yêu cầu bắt buộc đó là giáo viên phải đọc và hiểu tất cả các đoạn code của học sinh, nhưng công việc này lại tốn quá nhiều thời gian. Nếu bỏ qua các công việc như vậy thì người dạy không thể theo dõi được quá trình học tập của người học. Về phía người học, yêu cầu đặt ra là phải tiến bộ theo thời gian, nắm vững lý thuyết và thành thạo kỹ năng lập trình. Những không phải lúc nào gặp khó khăn người học đều có sự hỗ trợ kịp thời từ giảng viên. Họ có thể nhờ sự giúp đỡ từ đồng nghiệp, bạn bè, nhưng những người được nhờ giúp đỡ chưa chắc đã đủ trình độ, kinh nghiệm hoặc thời gian để ngồi bên cạnh giúp đỡ người học khi cần.\\

Để giảm bớt những khó khăn nêu trên, một công cụ hỗ trợ giáo viên và học sinh hiệu quả hơn, tiết kiệm thời gian hơn đó là một công cụ tự động hóa (có thể một phần) việc đánh giá kết quả lập trình của học sinh, cũng như hỗ trợ theo giõi sự tiến bộ của học sinh. Công cụ tự động hóa này sẽ tính toán, định lượng tỷ lệ chính xác sự tương tự về hành vi giữa chương trình của người học và chương trình của người dạy đưa ra trước đó. Dựa trên kết quả, giáo viên sẽ đánh giá được kỹ năng lập trình của học sinh, sự giống nhau giữa hai chương trình càng cao thì tỷ lệ độ tương tự càng cao, điểm số càng cao. Nếu điểm số thấp, người học có thể quay lại kiểm tra để viết mã chương trình đúng hơn, hạn chế được nguy cơ tìm ẩn trong cách viết chương trình của người học.\\

Những đánh giá này có thể thực hiện được nếu ta đo được độ tương tự giữa các chương trình có độ chính xác cao. Đề tài “Độ tương tự về hành vi của các chương trình và làm thực nghiệm” với mục đích sẽ giải quyết các vấn đề nếu trên.

\section{Đối tượng, phạm vi, phương pháp nghiên cứu}

\subsection*{Mục tiêu nghiên cứu}

\subsubsection*{Mục tiêu nghiên cứu chính}
Đánh giá độ tương tự về hành vi của các chương trình

\subsubsection*{Mục tiêu nghiên cứu cụ thể}

- Tìm hiểu sự tương tự ngữ nghĩa của chương trình

- Tìm hiểu kỹ thuật, công cụ sinh Test Case tự động

- Phân tích các độ đo và áp dụng kỹ thuật sinh Test Case tự động trên các độ đo

- Tìm cách kết hợp các độ đo với nhau

- Tìm một số ứng dụng của độ đo, chọn một ứng dụng để làm thực nghiệm, đánh giá kết quả thực nghiệm

\subsection*{Đối tượng, phạm vi nghiên cứu}

\subsubsection*{Đối tượng nghiên cứu}

- Kỹ thuật sinh Test Case

- Độ đo tương tự hành vi

- Một số ứng dụng của độ đo

\subsubsection*{Phạm vi nghiên cứu}

- Đo độ tương tự hành vi dựa vào Test Case

- Thực nghiệm, đánh giá trên các chương trình C Sharp



\subsection*{Phương pháp nghiên cứu, thực nghiệm}

\subsubsection*{Nghiên cứu lý thuyết}

\begin{itemize}
\item Độ tương tự hành vi
\item Một số kỹ thuật sinh Test Case tự động
\item Độ đo tương tự hành vi dựa trên Test Case
\item So sánh, kết hợp các độ đo
\end{itemize}


\subsubsection*{Thực nghiệm}

\begin{itemize}
\item Tiến hành cài kỹ thuật đo độ tương tự hành vi
\item Thực nghiệm trên dữ liệu thực của CodeHunt
\item Phân tích, đánh giá dựa trên kết quả thực nghiệm
\end{itemize}


\subsection{Lý do chọn đề tài, ngữ cảnh bài toán}
		
Trong những năm gần đây, xu hướng đào tạo lập trình viên nói riêng và
công nghệ phần mềm nói chung đang ngày càng trở nên phổ biến. Các
trường đại học và một số trung tâm đào tạo đã cho ra đời nhiều chương
trình đào tạo phong phú về nội dung, đa dạng về hình thức và thu hút
được nhiều sự quan tâm.
	
Trong mỗi khóa học, số lượng học viên thường có hằng trăm, hằng ngàn
người tham gia, nhưng chỉ một vài giáo viên giảng dạy. Trong đó, chất
lượng việc quản lý, truyền đạt nội dung của giáo viên và mức độ hiểu
biết, nắm bắt nội dung chương trình học của học viên là yêu cầu tất cả
các khóa học cần đạt được. Để đạt được điều đó, một yêu cầu bắt buộc
đó là giáo viên phải đọc và hiểu tất cả các đoạn code của học sinh,
nhưng công việc này lại tốn quá nhiều thời gian. Nếu bỏ qua hoặc trì
hoãn các công việc như vậy thì người dạy không thể theo dõi được quá
trình học tập của người học. Về phía người học, yêu cầu đặt ra là phải
tiến bộ theo thời gian, nắm vững lý thuyết và thành thạo kỹ năng lập
trình. Những không phải lúc nào gặp khó khăn người học đều có sự hỗ
trợ kịp thời từ giảng viên. Họ có thể nhờ sự giúp đỡ từ đồng nghiệp,
bạn bè, nhưng những người được nhờ hiups đỡ chưa chắc đã đủ trình độ,
kinh nghiệm hoặc thời gian để ngồi bên cạnh giúp đỡ người học khi cần.
	
Để giảm bớt những khó khăn nêu trên, một công cụ hỗ trợ quá trình
giảng dạy và học tập của giáo viên và học sinh hiệu quả hơn, tiết kiệm
thời gian hơn đó là một công cụ tự động hóa (có thể một phần) việc
đánh giá kết quả lập trình của học sinh, cũng như hỗ trợ theo giõi sự
tiến bộ của học sinh. Công cụ tự động hóa này sẽ tính toán, định lượng
tỷ lệ chính xác sự tương tự về hành vi giữa chương trình của người học
và chương trình của người dạy đưa ra trước đó. Dựa trên kết quả, giáo
viên sẽ đánh giá được kỹ năng lập trình của học sinh, sự giống nhau
giữa hai chương trình càng cao thì tỷ lệ độ tương tự càng cao, điểm số
càng cao. Nếu điểm số thấp, người học có thể quay lại kiểm tra để viết
mã chương trình đúng hơn, hạn chế được nguy cơ tìm ẩn trong cách viết
chương trình của người học.
	
Những đánh giá này có thể thực hiện được nếu ta đo được độ tương tự
giữa các chương trình có độ chính xác cao. Đề tài “Độ tương tự về hành
vi của các chương trình và làm thực nghiệm” với mục đích sẽ giải quyết
các vấn đề nếu trên.

\section{Những nghiên cứu có liên quan}

\todo{Đọc related works trong bài báo, trình bày lại, nhớ tham chiếu}


\section*{Tổng kết chương}

\todo{Viết tổng kết mỗi chương}


%%% Local Variables:
%%% mode: latex
%%% TeX-master: "Main"
%%% End:
