\documentclass[12pt,a4paper,oneside,openright]{memoir}

\usepackage[left=3.5cm,right=2cm,top=3.5cm,bottom=3cm]{geometry}
\usepackage[utf8]{vietnam}
\setlength{\parindent}{0pt} % Ngăn thụt vào đầu dòng ở đầu dòng mỗi đoạn
\setlength{\parskip}{0.5em}
\renewcommand{\baselinestretch}{1.0}
\usepackage{amsmath}
\usepackage{amsfonts}
\usepackage{amssymb}
\usepackage{fancyhdr}
\pagestyle{fancy}
\fancyhf{}
\chead{\thepage}
\renewcommand{\headrulewidth}{0.1pt}
\usepackage{pdfpages}
%-----------------------
\graphicspath{ {images/} }
\usepackage{lipsum}
\usepackage{listings}
\usepackage{hyperref}
\usepackage{algorithm }
\usepackage[noend]{algpseudocode}
\usepackage{graphicx}
\graphicspath{ {images/} }
\lstset{inputpath=Filecodes}
\newtheorem{definition}{Định nghĩa}
\renewcommand{\lstlistingname}{Mã lệnh}
%--------------------------------------------
\usepackage{tabu}
\usepackage{color}
\definecolor{dkgreen}{rgb}{0,0.6,0}
\definecolor{gray}{rgb}{0.5,0.5,0.5}
\definecolor{mauve}{rgb}{0.58,0,0.82}
\lstset{inputpath=Filecodes}
\lstset{frame=tb,
	language=[Sharp]C,
	aboveskip=3mm,
	belowskip=3mm,
	showstringspaces=false,
	columns=flexible,
	basicstyle={\small\ttfamily},
	numbers=none,
	numberstyle=\tiny\color{gray},
	keywordstyle=\color{blue},
	commentstyle=\color{dkgreen},
	stringstyle=\color{mauve},
	breaklines=true,
	breakatwhitespace=true,
	tabsize=3
}

%\usepackage[utf8]{vietnam}
%\usepackage{pdfpages}
%\usepackage{todonotes}
%\usepackage{hyperref}
%
%\usepackage{amsmath}
%\usepackage{amssymb}
%\usepackage{algorithm}
%\usepackage[noend]{algpseudocode}
%
%\usepackage{graphicx}
%\graphicspath{ {images/} }
%\usepackage{lipsum}
%\usepackage[left=3.5cm,right=2cm,top=3.5cm,bottom=3cm]{geometry}
%
%\usepackage{fancyhdr}
%\pagestyle{fancy}
%\fancyhf{}
%\chead{\thepage}
%\renewcommand{\headrulewidth}{0.1pt}
%
%
%\setlength{\parskip}{0.5em} % Khoảng cách giữa các Paragraph
%
%\usepackage{mathpazo}
%\usepackage{lipsum}
%
%% Cấu hình file --> để hiển nhập code c# 
%\usepackage{listings}
%\usepackage{color}
%\definecolor{dkgreen}{rgb}{0,0.6,0}
%\definecolor{gray}{rgb}{0.5,0.5,0.5}
%\definecolor{mauve}{rgb}{0.58,0,0.82}
%\lstset{inputpath=Filecodes}
%\lstset{frame=tb,
%	language=[Sharp]C,
%	aboveskip=3mm,
%	belowskip=3mm,
%	showstringspaces=false,
%	columns=flexible,
%	basicstyle={\small\ttfamily},
%	numbers=none,
%	numberstyle=\tiny\color{gray},
%	keywordstyle=\color{blue},
%	commentstyle=\color{dkgreen},
%	stringstyle=\color{mauve},
%	breaklines=true,
%	breakatwhitespace=true,
%	tabsize=3
%}
%
%% Cấu hình table
%\usepackage{tabu}
%
%\newtheorem{definition}{Định nghĩa}
%
%\renewcommand{\lstlistingname}{Mã lệnh}


\title{Độ tương tự hành vi của chương trình \\ và thực nghiệm}
\author{Đỗ Đăng Khoa}

\begin{document}

\includepdf[pages=-]{trangbia.pdf}

% \pagenumbering{gobble} % Trước trang không đánh số thứ tự 

% \pagenumbering{roman} % Trước trang đánh số la mã
% \setcounter{page}{1}

\chapter*{LỜI CAM ĐOAN}
\input{Loicamdoan}

\chapter*{LỜI CẢM ƠN}

Tôi xin chân thành cảm ơn sự hướng dẫn, chỉ dạy và giúp đỡ tận tình của các thầy cô giảng dạy sau đại học - Trường đại học Quy Nhơn.

Đặc biệt, tôi cảm ơn thầy TS.Phạm Văn Việt, giảng viên bộ môn Công nghệ phần mềm, khoa Công nghệ thông tin, Trường Đại học Quy Nhơn đã tận tình hướng dẫn truyền đạt những kiến thức và kinh nghiệm quý báu đễ giúp tôi có đầy đủ kiến thức và nghị lực hoàn thành luận văn này.

Và tôi xin cảm ơn bạn bè, đồng nghiệp và những người thân trong gia đình đã tin yêu, động viên giúp tôi thêm nghị lực trong quá trình học tập và nghiên cứu.

Mặc dù đã cố gắng rất nhiều trong việc thực hiện luận văn, song với thời gian có hạn, nên luận văn không thể tránh khỏi những thiếu sót và chưa hoàn chỉnh. Tôi rất mong nhận được ý kiến đóng góp của quý Thầy Cô và các bạn.

Một lần nữa, tôi xin chân thành cảm ơn!.


\begin{tabu} to 1.0 \textwidth {  X[c] X[c]  X[c]  X[c]  }

	 & &  & \textbf{HỌC VIÊN} \\
	 \\ \\ \\ \\ \\ \\ \\ \\ \\ \\ \\ 
	 & &  & \textbf{Đỗ Đăng Khoa}  \\
         
       \end{tabu}


       
%%% Local Variables:
%%% mode: latex
%%% TeX-master: "Main"
%%% End:


\includepdf[pages=-]{Danhmuckyhieu.pdf}

\newpage
\tableofcontents

%======================================================================
% \pagenumbering{arabic} % Trước trang đánh số thứ tự thêm 2 dòng này
% \setcounter{page}{1}   % Trước trang đánh số thứ tự thêm 2 dòng này - số trang bắt đầu bằng 3

\includepdf[pages=-]{Tomtat.pdf}

\chapter{GIỚI THIỆU}
\newpage
\chapter{GIỚI THIỆU}
\section{Giới thiệu chung}
Hiện nay, ngành Công nghệ thông tin với các chương trình đào tạo lập trình Online, kỹ sư phần mềm... đang trở nên rất phổ biến. Trên thế giới cũng có nhiều chương trình đào tạo nổi tiếng như Massive Open Online Courses (MOOC), edX, Coursera, Udacity thu hút hằng ngàn sinh viên theo học. Một số chương trình học lập trình online như Pex4Fun hay Code Hunt là một nền tảng học lập trình online thông qua trò chơi.\\

Những lớp học như thế này thường đặt ra một số thách thức, làm thế nào để kiểm soát được chất lượng học tập của người học. Trong khi các lớp học có hằng trăm người tham gia, nhưng thành viên tham giảng dạy chỉ vài người trong một lớp học. Hằng ngày, người giáo viên phải thường xuyên kiểm tra, nhắc nhở và phải đọc, nghiên cứu giúp đỡ cho học viên. Chỉ riêng việc đọc và hiểu code do học viên viết ra đã tốn quá nhiều thời gian của người dạy, nếu như bỏ qua thì khó có thể đánh giá được chất lượng của người học.\\

Để giảm bớt những vất vả, khó khăn của người dạy và người học. Một công cụ hỗ trợ, tự động đánh giá hành vi chương trình của người học sẽ giúp tiết kiệm được thời gian, giúp cho giáo viên việc quản lý chất lượng học tập của học viên được tốt hơn. Ví dụ, công cụ sẽ tự động đánh giá hành vi, so sánh hành vi chương trình của người học viết với hành vi chương trình của giáo viên. Nếu hành vi của hai chương trình có tỷ lệ giống nhau càng cao thì điểm số cho chương trình của người học càng cao. Ngược lại, nếu điểm số thấp người dạy và học sẽ tìm hiểu nguyên nhân vì sao và có hướng khác phục những hạn chế mà người học đang gặp phải. Đây cũng là một cách giúp cho người học không đi lệch khỏi kiến thức của chương trình đào tạo, tiết kiệm được thời gian cả người dạy và người học.\\

Ý tưởng trong việc đánh giá độ tương tự hành vi của hai chương trình đó là tính toán tìm ra một miền giá trị đầu vào chung cho cả hai chương trình. Đưa từng giá trị đầu vào chạy đồng thời trên cả hai chương trình và so sánh kết quả đầu ra của hai chương trình. Tuy nhiên, việc này sẽ không đạt, không khả thi, vì trong những trường hợp chương trình có miền giá trị đầu vào lớn hoặc vô hạn. Vì vậy, để đánh giá độ tương tự hành vi của hai chương trình khả thi hơn nếu chúng ta chỉ sử dụng một sô giá trị đầu vào đại diện cho toàn bộ miền giá trị đầu vào chung của hai chương trình.\\

Kỹ thuật Dynamic Symbolic Execution (DSE) là một kỹ thuật thu thập các ràng buộc từ các nhánh của chương trình, phủ nhận lại có hệ thống một phần các ràng buộc để tạo ra giá trị đầu vào cho một chương trình. Hiện nay, đã có nhiều công cụ sử dụng kỹ thuật DSE để giải quyết các ràng buộc một cách mạnh mẽ, tạo ra các giá trị đầu vào tin cậy có độ phủ cao.

\begin{center}
\begin{tabular}  {|c|c|c|} 
	\hline 
	\textbf{Tên Công cụ} & \textbf{Ngôn ngữ} & \textbf{Url} \\ 
	\hline 
	KLEE & LLVM & klee.github.io/ \\ 
	\hline 
	JPF	 & Java	& babelfish.arc.nasa.gov/trac/jpf \\
	\hline 
	jCUTE &	Java &	github.com/osl/jcute \\
	\hline 
	janala2	 & Java &	github.com/ksen007/janala2 \\
	\hline 
	JBSE	& Java	 & github.com/pietrobraione/jbse \\
	\hline 
	KeY &	Java &	www.key-project.org/ \\	
	\hline 
	Mayhem & 	Binary &	forallsecure.com/mayhem.html \\
	\hline 
	Otter &	C	& bitbucket.org/khooyp/otter/overview \\
	\hline 
	Rubyx & 	Ruby &	www.cs.umd.edu/~avik/papers/ssarorwa.pdf \\
	\hline 
	Pex	& .NET Framework	 & research.microsoft.com/en-us/projects/pex/ \\
	\hline 
	Jalangi2 &	JavaScript &	github.com/Samsung/jalangi2 \\
	\hline 
	Kite &	LLVM &	www.cs.ubc.ca/labs/isd/Projects/Kite/ \\
	\hline 
	pysymemu &	x86-64 / Native	 &github.com/feliam/pysymemu/ \\
	\hline 
	Triton	& x86 and x86-64 &	triton.quarkslab.com \\	
	\hline 
	BE-PUM &	x86	 & https://github.com/NMHai/BE-PUM	 \\	
	\hline

\end{tabular} 
\end{center}


	

\chapter{KIẾN THỨC CƠ SỞ}
\newpage
\chapter{KIẾN THỨC CƠ SỞ}

\section{Một số khái niệm, định nghĩa}
\subsection{Kỹ thuật Dynamic symbolic execution (DSE) }
\subsubsection{Định nghĩa}
Là một kỹ thuật thu thập các ràng buộc từ các nhánh của chương trình và phủ nhận một phần các ràng buộc để tạo ra dữ liệu đầu vào của chương trình và có độ phủ cao


\section{Kỹ thuật lập trình trên C sharp}


	


\chapter{ĐO ĐỘ TƯƠNG TỰ VỀ HÀNH VI GIỮA CÁC CHƯƠNG TRÌNH}
Chương này trình bày một số định nghĩa liên quan đến hành vi chương trình, các phép đo độ tương tự hành vi của chương trình và tiêu chí đánh giá các phép đo này.

\section{Hành vi của chương trình }
Để định lượng hai chương trình tương tự nhau, chúng ta nghiên cứu các định nghĩa liên quan hành vi của hai chương trình như sau:
	
\subsection{Thực thi chương trình}
\begin{definition}\label{def:progexe}
Cho $P$ là một chương trình, $I$ là tập hợp các trị đầu vào của $P$ và $O$ là tập hợp các giá trị đầu ra của $P$. Thực thi chương trình P là ánh xạ $exec: P \times I \rightarrow O$. Với giá trị đầu vào $i \in I$, sau khi thực thi $P$ trên $i$ ta có giá trị đầu ra tương ứng $o \in O$ và ký hiệu $o = exec(P, i)$.  
\end{definition}

\subsection{Độ tương đương về hành vi (Behavioral Equivalence)}

Dựa trên định nghĩa về thực thi chương trình, chúng ta tìm hiểu thế nào là độ tương đương về hành vi giữa hai chương trình thông qua hai chương trình minh họa sau:

\begin{minipage}[t]{0.45\linewidth}
	\lstinputlisting[caption = {Switch...Case}]{SwitchCase.cs}
\end{minipage}%
\hfill\vrule\hfill
\begin{minipage}[t]{0.45\linewidth}
	\lstinputlisting[caption = {If...Else}]{IfElse.cs}
\end{minipage}%

Mã lệnh $3.1$ và $3.2$ có tham số đầu vào cùng kiểu giá trị $int$. Mã lệnh $3.1$ sử dụng cấu trúc $\textit{\textbf{switch...case}}$, mã lệnh $3.2$ sử dụng cấu trúc $\textit{\textbf{if...else}}$ để kiểm tra giá trị đầu vào $x$. Mặc dù cú pháp sử dụng trong hai chương trình là khác nhau nhưng cách thức xử lý trả về kết quả $y$ là như nhau. Từ đó, chúng ta có thể định nghĩa thế nào là độ tương đương hành vi giữa hai chương trình như sau:

\begin{definition}[Độ tương đương về hành vi]
 Cho $P_{1}$ và $P_{2}$ là hai chương trình có cùng miền các giá trị đầu vào $I$. Hai chương trình này được gọi là tương đương khi và chỉ khi thực thi của chúng giống nhau trên mọi giá trị đầu vào trên $I$, ký hiệu là exec($P_{1}, I$) = exec($P_{2}, I$). 
\end{definition}	
	
\subsection{Sự khác biệt về hành vi (Behavioral Difference) }
Để tìm hiểu sự khác biệt về hành vi của hai chương trình, chúng ta tìm hiểu hai mã lệnh sau:

\begin{minipage}[t]{0.45\linewidth}
	\lstinputlisting[caption = {Chương trình P\_1}]{Khac_biet_HV_1.cs}
\end{minipage}%
\hfill\vrule\hfill
\begin{minipage}[t]{0.45\linewidth}
	\lstinputlisting[caption = {Chương trình P\_2}]{Khac_biet_HV_2.cs}
\end{minipage}%

Mã lệnh $3.3$ và Mã lệnh $3.4$ của Chương trình $P_{1}$ và $P_{2}$, cả hai Chương trình có miền giá trị đầu vào cùng kiểu $int$, giá trị trả về của Chương trình $P_{1}$ là $x - 10$, giá trị trả về của Chương tình $P_{2}$ là $x + 10$. Với mọi giá trị của $x$ được thực thi trên cả hai chương trình $P_{1}$ và $P_{2}$ kết quả trả về sẽ không giống nhau. Mặc dù cả hai Chương trình $P_{1}$ và $P_{2}$ có miền giá trị đầu vào như nhau, nhưng hành vi của hai Chương trình hoàn toàn khác nhau. Qua đó, chúng ta có thể định nghĩa sự khác biệt về hành vi như sau:

\begin{definition}[Sự khác biệt hành vi]
Cho $P_{1}$ và $P_{2}$ là hai chương trình có cùng một miền các giá trị đầu vào $I$. Hai chương trình này được xem là có sự khác biêt về hành vi khi và chỉ khi thực thi của chúng khác nhau trên mọi giá trị đầu vào $I$, ký hiệu là $exec(P_{1}, I) \neq exec(P_{2}, i)$.
\end{definition}

\subsection{Độ tương tự hành vi (Behavioral Similarity)}
Để hiểu thế nào là tương tự hành vi, chúng ta phân tích hai Mã lệnh sau:

\begin{minipage}[t]{0.45\linewidth}
	\lstinputlisting[caption = {Chương trình $P_{1}$}]{TuongTu_HV_1.cs}
\end{minipage}%
\hfill\vrule\hfill
\begin{minipage}[t]{0.45\linewidth}
	\lstinputlisting[caption = {Chương trình $P_{2}$}]{TuongTu_HV_2.cs}
\end{minipage}%

Với Mã lệnh $3.5$ và Mã lệnh $3.6$ của hai Chương trình $P_{1}$ và $P_{2}$, chúng ta thấy cả hai Chương trình có giá trị đầu vào cùng kiểu dữ liệu là $int$, nếu giá trị đầu vào của biến $x$ nằm trong khoảng $0$ đến $100$ thì giá trị trả về của cả hai Chương trình đều bằng nhau là $x+10$. Ngươc lại, giá trị đầu vào của biến $x$ nằm ngoài khoảng $0$ đến $100$, thì giá trị trả về của Chương trình $P_{1}$ là $x$ và giá trị trả về của Chương trình $P_{2}$ là $-1$. Hai Chương trình $P_{1}$ và $P_{2}$ tuy có kiểu dữ liệu đầu vào như nhau nhưng kết quả đầu ra có thể giống nhau hoặc khác nhau tùy theo giá trị đầu vào của biến $x$. Dựa trên kết quả phân tích, chúng ta định nghĩa độ tương tự hành vi của Chương trình như sau:

\begin{definition}[Độ tương tự hành vi]
Cho $P_{1}$ và $P_{2}$ là hai chương trình có cùng miền giá trị đầu vào $I$, và $I_{s}$ là tập con của $I$. Hai Chương trình được xem là tương tự hành vi khi thực thi chúng giống nhau trên mọi giá trị đầu vào $I_{s}$, ký hiệu  $exec(P_{1}, I_{s}) = exec(P_{2}, I_{s})$ và khác nhau $\forall j \in I \setminus I_{s}$, ký hiệu $exec(P_{1}, j) \neq exec(P_{2}, j)$
\end{definition}

\section{Một số phép đo độ tương tự hành vi}
Để đo độ tương tự về hành vi giữa hai Chương trình, chúng ta có thể chạy từng giá trị đầu vào trong miền giá trị đầu vào của hai Chương trình. Tỷ lệ giữa số lượng đầu vào thử nghiệm khi thực thi trên cả hai Chương trình cho kết quả đầu ra giống nhau trên tổng số lượng đầu vào được thử nghiệm là độ tương tự hành vi giữa hai Chương trình. Dựa trên cách tính tỷ lệ kết quả đầu ra của hai Chương trình, chúng ta có một số phép đo độ tương tự hành vi như sau:

\subsection{Phép đo lấy mẫu ngẫu nhiên (RS)}
Kỹ thuật của phép đo \textbf{RS} là thực hiện lấy ngẫu nhiên giá trị đầu vào trên miền giá trị đầu vào của hai Chương trình. Thực thi cả hai Chương trình trên từng giá trị đầu vào, tiến hành so sánh giá trị kết quả đầu ra của cả hai Chương trình. Tỷ lệ giữa tổng số mẫu đầu vào khi thực thi những mẫu này hai chương trình cho kết quả đầu ra có giá trị giống nhau, trên tổng số mẫu đầu vào được thử nghiệm là kết quả cho phép đo RS. Từ đó, chúng ta có định nghĩa phép đo \textbf{RS} như sau:
 
\begin{definition}[Phép đo RS]
Cho $P_{1}$ và $P_{2}$ là hai Chương trình có cùng miền giá trị đầu vào $I$, $I_{s}$ là tập con ngẫu nhiên của $I$, $I_{a}$ là tập con $I_{s}$. Hai Chương trình thực thi giống nhau với $\forall i \in I_{a}$, ký hiệu $exec(P_{1}, i) = exec(P_{2}, i)$ và hai Chương trình thực thi khác nhau với $\forall j \in I_{s} \setminus I_{a}$, ký hiệu $exec(P_{1}, j) \neq exec(P_{2}, j)$. Chỉ số phép đo \textbf{RS} được định nghĩa là $M_{RS}(P_{1}, P_{2}) = \left|I_{a}\right| \diagup \left|I_{s}\right| $.
\end{definition}

Phép đo \textbf{RS} là một một phép đo đơn giản và hiệu quả để tính độ tương tự của hành vi. Khi miền giá trị của tham số đầu vào rất lớn hoặc vô hạn, phép đo \textbf{RS} thực hiện lấy mẫu ngẫu nhiên để tính toán độ tương tự của hành vi, kết quả đầu ra tương đối tốt và hợp lý so với hành vi thực tế của chương trình. Phép đo \textbf{RS} xử lý, tính toán độ tương tự hành vi dưới dạng hộp đen và không phân tích chương trình để tạo thử nghiệm, nên tốc độ xử lý nhanh và chiếm ít tài nguyên. Mặc khác, phép đo \textbf{RS} không phân tích chương trình để tạo đầu vào thử nghiệm nên phép đo \textbf{RS} có thể bỏ qua một vài tham số đầu vào thử nghiệm có thể được sử dụng để thực thi một số nhánh khác nhau giữa hai chương trình. Vì vậy, phép đo \textbf{RS} không phân biệt được các chương trình có một số hành vi khác nhau. Chúng ta phân tích Mã lệnh $3.7$ và $3.8$ sau để thấy được hạn chế của phép đo \textbf{RS}:

\begin{minipage}[t]{0.45\linewidth}
	\begin{lstlisting}[caption={Chương trình $P_{1}$}, label={Script}]
	public static int Y(string x) {
		if (x == "XYZ") return 0;
		if (x == "ABC")	return 1	
		return -1;
	}
	\end{lstlisting}
\end{minipage}%
\hfill\vrule\hfill
\begin{minipage}[t]{0.45\linewidth}
	\begin{lstlisting}[caption={Chương trình $P_{2}$}, label={Script}]
	public static int Y(string x) {
		if (x == "ABC")	return 1	
		return -1;
	}
	\end{lstlisting}
\end{minipage}%

Chúng ta thấy đoạn Mã lệnh $3.7$ và $3.8$ của hai Chương trình $P_{1}$ và $P_{2}$ có cùng miền giá trị đầu vào là $string x$, cấu trúc mã lệnh hai chương trình gần như nhau. Nhưng Chương trình $P_{1}$ khác với Chương trình $P_{2}$ đó là sẽ trả kết quả về $0$ nếu tham số đầu vào có giá trị là $XYZ$. Tỷ lệ phép đo \textbf{RS} lấy ngẫu nhiên giá trị đầu vào $x$ trên miền giá trị đầu vào của hai chương trình có giá trị bằng $XYZ$ là rất thấp, vì vậy khả năng câu lệnh $if (x == "XYZ") return 0;$ của Chương trình $P_{1}$ có thể sẽ không được thực thi nên kết quả của phép đo \textbf{RS} sẽ ở mức tương đối so với hành vi thực tế của chương trình.
	
\subsection{Phép đo tượng trưng trên một chương trình Single Program Symbol Execution (SSE)}
Phép đo \textbf{SSE} là một phép đo dựa trên số lượng các nhánh đường đi của Chương trình mẫu, mỗi nhánh đường đi của Chương trình mẫu được xem là một hành vi của chương trình. Nếu chọn một giá trị đầu vào thử nghiệm cho một nhánh đường đi trong Chương trình thì các giá trị đầu vào thử nghiệm này sẽ khám phá hết các hành vi trong Chương trình mẫu. Do vậy, số phần tử trong tập các giá trị đầu vào thử nghiệm của phép đo \textbf{SSE} sẽ nhỏ hơn tập các giá trị đầu vào thử nghiệm được chọn theo phương pháp lấy ngẫu nhiên giá trị đầu vào. 

Để tính độ tương tự hành vi của hai chương trình với phép đo \textbf{SSE}, chúng ta chọn Chương trình mẫu làm Chương trình tham chiếu và áp dụng kỹ thuật \textbf{DSE} để tạo ra các đầu vào thử nghiệm dựa trên Chương trình tham chiếu. Sau đó thực thi cả hai chương trình dựa trên các giá trị đầu vào thử nghiệm. Tỷ lệ số lượng các kết quả đầu ra giống nhau của cả hai chương trình trên tổng số các giá trị đầu vào thử nghiệm của Chương trình tham chiếu là kết quả của phép đo \textbf{SSE}. Qua đó, chúng ta có định nghĩa phép đo \textbf{SSE} như sau:

\begin{definition}
 Cho $P_{1}$ và $P_{2}$ là hai chương trình có cùng miền giá trị đầu vào $I$, Chương trình $P_{1}$ là Chương trình tham chiếu, $I_{s}$ là tập các giá trị đầu vào được tạo bởi DSE trên chương trình $P_{1}$, và $I_{a}$ là tập con $I_{s}$. Hai Chương trình thực thi giống nhau với $\forall i \in I_{a}$, ký hiệu $exec(P_{1}, i) = exec(P_{2}, i)$ và hai Chương trình thực thi khác nhau với $\forall j \in I_{s} \setminus I_{a}$, ký hiệu $exec(P_{1}, j) \neq exec(P_{2}, j)$. Chỉ số phép đo \textbf{SSE} được định nghĩa là $M_{SSE}(P_{1}, P_{2}) = \left|I_{a}\right| \diagup \left|I_{s}\right| $.
\end{definition}

Ngược lại với phép đo RS, phép đo SSE khám phá những đường đi khả thi khác nhau trong chương chình tham chiếu để tạo dữ liệu đầu vào của chương trình. Do đó, các đầu vào thử nghiệm này sẽ thực thi hết các đường đi của chương trình tham chiếu và có khả năng phát hiện được những chương trình cần tính có những hành vi khác so với Chương trình tham chiếu. Những phép đo SSE vẫn còn hạn chế, đó là phép đo SSE không xem xét đường đi của Chương trình cần phân tích để tạo các giá trị đầu vào thử nghiệm mà chỉ dựa vào các đầu vào thử nghiệm được phân tích từ Chương trình tham chiếu. Các đầu vào thử nghiệm này không nắm bắt được hết các hành vi của Chương trình cần phân tích, Chương trình cần phân tích có thể sẽ có những hành vi khác so với Chương trình tham chiếu. Một số chương trình có thể có những vòng lập vô hạn phụ thuộc vào giá trị đầu vào nên SSE không thể liệt kê được tất cả các đường dẫn của chương trình. Chúng ta xem xét và phân tích 2 đoạn Mã lệnh $3.9$ và $3.10$ để thấy được hạn chế của phép đo SSE như sau:

\begin{minipage}[t]{0.45\linewidth}
	\begin{lstlisting}[caption={Chương trình $P_{1}$}, label={Script}]
	public static int Y(int x) {
	    if(x>0 && x<100){
            return x - 10;
	    }
	    else {
        	return x; 
	    }
    }
	\end{lstlisting}
\end{minipage}%
\hfill\vrule\hfill
\begin{minipage}[t]{0.45\linewidth}
	\begin{lstlisting}[caption={Chương trình $P_{2}$}, label={Script}]
	public static int Y(int x){
	    if(x>=0 && x<100){
        	return x - 10;
	    }
	    else {
        	return x; 
	    }
    }
	\end{lstlisting}
\end{minipage}%

Hai đoạn Mã lệnh $3.9$ và Mã lệnh $3.10$ của hai Chương trình $P_{1}$ và $P_{2}$, chọn Chương trình $P_{1}$ làm Chương trình tham chiếu, sử dụng kỹ thuật \textbf{DSE} để phân tích chương trình $P_{1}$ ta được tập các giá trị đầu vào thử nghiệm là ${(0, 1)}$. Trong khi đó, phân tích Chương trình $P_{2}$ chúng ta được tập các giá trị đầu vào thử nghiệm của Chương trình $P_{2}$  là ${(-1, 0, 1)}$. Do đó, chúng ta thấy tập giá trị đầu vào thử nghiệm do phép đo \textbf{SSE} tạo ra thiếu giá trị đầu thử nghiệm $-1$ để có thể thực thi hết các đường đi của Chương trình $P_{2}$.

\subsection{Kỹ thuật thực thi chương trình kết hợp Paired Program Symbolic Execution (PSE)}
Để giải quyết giới hạn của phép đo \textbf{SSE} khi tạo ra tập các giá trị đầu vào thử nghiệm không thực thi hết các các đi của Chương trình cần phân tích. Phép đo \textbf{PSE} giải quyết giới hạn của phép đo \textbf{SSE} bằng cách tạo một Chương trình kết hợp giữa Chương trình cần phân tích với Chương trình tham chiếu. Dựa trên Chương trình kết hợp sử dụng kỹ thuật \textbf{DSE} để tạo ra đầu vào thử nghiệm cho cả hai chương trình, các đầu vào thử nghiệm này bao gồm các đầu vào thử nghiệm đúng và không đúng. Các đầu vào thử nghiệm đúng là những giá trị khi thực thi trên cả hai chương trình sẽ cho kết quả đầu ra như nhau, ngược lại các đầu vào thử nghiệm không đúng là những giá trị khi thực thi trên cả hai chương trình sẽ cho kết quả khác nhau. Do đó, phép đo \textbf{PSE} được tính bằng tỷ lệ các giá trị đầu vào thử nghiệm đúng trên tổng số các giá trị đầu vào được thử nghiệm. 

\begin{lstlisting}[caption={Chương trình $P_{3}$, label={Script}]
	public int $P_{3}$ (int number) {
		if(Program1(args) == Program2(args))
			return 1;
		return 0;
	}
\end{lstlisting}

Dựa trên cách thức hoạt động của phép đo \textbf{PSE}, chúng ta có định nghĩa phép đo \textbf{PSE} như sau:

\begin{definition}[Phép đo PSE]
Cho $P_{1}$ và $P_{2}$ là hai Chương trình có cùng miền giá trị đầu vào $I$. $P_{3}$ là Chương trình
kết hợp của $P_{1}$ và $P_{2}$, ký hiệu $(exec(P_{1}, I) =  exec(P_{2}, I))$.  $I_{s}$ là tập các giá trị đầu vào được tạo bởi \textbf{DSE} từ Chương trình $P_{3}$, $I_{a}$ là tập con $I_{s}$. Hai Chương trình $P_{1}$ và $P_{2}$ thực thi giống nhau khi Chương trình $P_{3}$ thực thi cho giá trị đúng với $\forall i \in I_{a}$, ký hiệu $exec(P_{3}, i) = T$, và $\nexists j \in I_{s} \backslash I_{a}$ thực thi Chương trình $P_{3}$ cho giá trị đúng. Chỉ số phép đo \textbf{PSE} được định nghĩa là $M_{PSE}(P_{1}, P_{2}) = \left|I_{a}\right| \diagup \left|I_{s}\right| $.
\end{definition}

Phép đo \textbf{PSE} đã cải thiện được hạn chế của phép đo \textbf{SSE} khi dữ liệu thử nghiệm được  tạo ra từ trên Chương trình kết hợp, tập dữ liệu thử nghiệm có khả năng thực thi hết các nhánh đường đi của Chương trình tham chiếu và Chương trình cần tính. Tuy nhiên, phép đo \textbf{PSE} cũng có hạn chế trong quá trình xử lý các vòng lặp lớn hoặc vô hạn. Để giảm bớt hạn chế này, chúng ta có thể giới hạn miền đầu vào hoặc đếm số vòng lặp của các Chương trình. Ngoài ra, phép đo \textbf{PSE} khám phá đường dẫn của Chương trình kết hợp nên quá trình xử lý sẽ tốn thời gian và tài nguyên hơn so với phép đo \textbf{SSE} khi chỉ khám phá đường dẫn của Chương trình tham chiếu.

\section{Tiêu chí đánh giá hiệu quả}
Để đánh giá độ hiệu quả của các phép đo, chúng ta có thể áp dụng những tiêu chí cơ bản sau:
\begin{itemize}
\item Tốc độ xử lý
\item Sử dụng tài nguyên
\item Độ phủ của dữ liệu thử
\item Kết quả đánh giá độ tương tự hành vi
\end{itemize}

Phép đo \textbf{RS} sử dụng kỹ thuật lấy ngẫu nhiên giá trị thử nghiệm trong miền giá trị đầu vào của cả hai Chương trình nên tốc độ xử lý của phép đo \textbf{RS} nhanh, đơn giản và sử dụng ít tài nguyên. Nhưng độ phủ dữ liệu thử nghiệm do phép đo \textbf{RS} tạo ra  không cao, không phủ hết các trường hợp có thể thực thi của chương trình nên kết quả đánh giá độ tương tự hành vi của Chương trình đạt mức tương đối so với hành vi thực tế.

Phép đo \textbf{SSE} là một phép đo cải tiến của phép đo \textbf{RS}, khi sử dụng kỹ thuật \textbf{DSE} dựa trên Chương trình tham chiếu để tạo các giá trị đầu vào thử nghiệm. Vì phải khám tất cả các nhánh đường đi của Chương trình tham chiếu nên tốc độ xử lý của phép đo \textbf{SSE} sẽ chậm và chiếm nhiều tài nguyên hơn phép đo \textbf{RS}. Tập dữ liệu đầu vào thử nghiệm được tạo bởi phép đo \textbf{SSE} có khả năng phủ tất cả các nhánh của Chương trình tham chiếu nên kết quả đánh giá độ tương tự hành vi của phép đo \textbf{SSE} sẽ chính xác hơn kết quả đánh giá độ tương tự hành vi của phép đo \textbf{RS}.

Phép đo \textbf{PSE} sử dụng kỹ thuật \textbf{DSE} khám phá tất cả các nhánh đường đi của Chương trình kết hợp để tạo ra tập dữ liệu đầu vào thử nghiệm chung cho cả hai chương trình. Vì vậy, phép đo \textbf{PSE} sẽ tốn nhiều thời gian thực thi chương trình và chiếm nhiều tài nguyên hơn phép đo \textbf{SSE}. Dữ liệu thử nghiệm của phép đo \textbf{PSE} sẽ có độ phủ cao hơn phép đo \textbf{SSE}, vì tất cả các giá trị đầu vào thử nghiệm có khả năng phủ tất cả các nhánh của Chương trình tham chiếu và Chương trình cần tình. Kết quả đánh giá độ tương tự hành vi của phép đo \textbf{PSE} sẽ chính xác hơn kết quả đánh giá độ tương tự của phép đo \textbf{SSE}.

\section*{Tổng kết chương}
Nội dung chính được trình bày trong chương này bao gồm những định nghĩa về thực thi chương trình, độ tương tự hành vi, sự khác biệt về hành vi độ tương tự về hành vi của chương trình. Mô tả, định nghĩa 3 kỹ thuật đo RS, SSE, PSE cũng như trình bày những ưu điểm, nhược điểm và hướng khắc phục của 3 kỹ thuật đo. Qua đó, giới thiệu một số tiêu chí đánh giá hiệu quả các kỹ thuật đo. 

\chapter{THỰC NGHIỆM, ĐÁNH GIÁ, KẾT LUẬN}
Chương này trình bày những nội dung như sau:
\begin{itemize}
	\item Dữ liệu sử dụng thực nghiệm
	\item Những công cụ sử dụng trong thực nghiệm
	\item Đánh giá kết quả thực nghiệm
	\item Khả năng ứng dụng và hướng phát triển của đề tài
	\item Kết luận
\end{itemize}

\section{Dữ liệu thực nghiệm}
\label{sec:data}
\begin{center}
	\begin{figure}[htp]
		\begin{center}
			\includegraphics[scale=.4]{codehunt1.png}
		\end{center}
		\caption{Giao diện viết chương trình}
		\label{refhinh1}
	\end{figure}
\end{center}

Code Hunt \cite{CodeHunt} là một nền tảng chơi game, được sử dụng cho các cuộc thi viết mã và thực hành các kỹ năng lập trình. Code Hunt dựa trên công cụ thực thi biểu tượng Pex. Mã Hunt là một nền tảng mã hóa trực tuyến, trong đó mỗi câu đố được trình bày với các trường hợp kiểm tra, không có đặc điểm kỹ thuật. Đầu tiên người chơi phải chọn câu hỏi và trả lời mã câu hỏi bằng cách viết một đoạn mã sao cho kết quả trùng với kết quả cử câu hỏi. Code Hunt đã được hơn 350.000 người chơi sử dụng tính đến tháng 8 năm 2016. Dữ liệu từ các cuộc thi gần đây đã được công khai và cho phép tải về tập dữ liệu này để phân tích và nghiên cứu trong cộng đồng giáo dục.


Tập dữ liệu Code Hunt chứa các chương trình do sinh viên trên toàn thế giới viết, với 250 người sử dụng, 24 câu hỏi và khoảng 13.000 chương trình được sinh viên thực hiện trên 2 ngôn ngữ là Java và C Scharp. Để có thể sử dụng tập dữ liệu Code Hunt cho đề tài của tôi, tôi đã thực hiện chuyển đổi những chương trình bằng ngôn ngữ Java thành ngôn ngữ C\#  bằng công cụ chuyển đổi của hãng Tangible Software Solutions và loại bỏ một số chương trình lỗi và không phù hợp.


\begin{center}
	\begin{figure}[htp]
		\begin{center}
			\includegraphics[scale=.4]{java-to-csharp-collections.png}
		\end{center}
		\caption{Chuyển đổi code Java sang C Scharp}
		\label{refhinh1}
	\end{figure}
\end{center}

\section{Công cụ dùng trong thực nghiệm}
%Phần này trình bày những công cụ được sử dụng để triển khai thực nghiệm như: công cụ sinh dữ liệu kiểm thử, môi trường lập trình, \dots
\subsection*{Microsoft Visual studio}
Microsoft Visual Studio là một môi trường phát triển tích hợp từ Microsoft. Nó được sử dụng để phát triển chương trình máy tính cho Microsoft Windows, cũng như các trang web, các ứng dụng web và các dịch vụ web. Visual Studio sử dụng nền tảng phát triển phần mềm của Microsoft như Windows API, Windows Forms, Windows Presentation Foundation, Windows Store và Microsoft Silverlight. Nó có thể sản xuất cả hai ngôn ngữ máy và mã số quản lý.

Visual Studio bao gồm một trình soạn thảo mã hỗ trợ IntelliSense cũng như cải tiến mã nguồn. Trình gỡ lỗi tích hợp hoạt động cả về trình gỡ lỗi mức độ mã nguồn và gỡ lỗi mức độ máy. Công cụ tích hợp khác bao gồm một mẫu thiết kế các hình thức xây dựng giao diện ứng dụng, thiết kế web, thiết kế lớp và thiết kế giản đồ cơ sở dữ liệu. Nó chấp nhận các plug-in nâng cao các chức năng ở hầu hết các cấp bao gồm thêm hỗ trợ cho các hệ thống quản lý phiên bản (như Subversion) và bổ sung thêm bộ công cụ mới như biên tập và thiết kế trực quan cho các miền ngôn ngữ cụ thể hoặc bộ công cụ dành cho các khía cạnh khác trong quy trình phát triển phần mềm.

\begin{center}
	\begin{figure}[htp]
		\begin{center}
			\includegraphics[scale=.4]{visualstudio.png}
		\end{center}
		\caption{Giao diện phần mềm Visual studio 2013}
		\label{refhinh1}
	\end{figure}
\end{center}

Visual Studio hỗ trợ nhiều ngôn ngữ lập trình khác nhau và cho phép trình biên tập mã và gỡ lỗi để hỗ trợ (mức độ khác nhau) hầu như mọi ngôn ngữ lập trình. Các ngôn ngữ tích hợp gồm có C,[1] C++ và C++/CLI (thông qua Visual C++), VB.NET (thông qua Visual Basic.NET), C thăng (thông qua Visual C thăng) và F thăng (như của Visual Studio 2010). Hỗ trợ cho các ngôn ngữ khác như J++/J thăng, Python và Ruby thông qua dịch vụ cài đặt riêng rẽ. Nó cũng hỗ trợ XML/XSLT, HTML/XHTML, JavaScript và CSS.

\subsection*{Ngôn ngữ lập trình C\# }
\subsubsection*{Tổng quan Ngôn ngữ C\# }
Quá trình dịch chương trình trong C\# là một ngôn ngữ lập trình hiện đại, được phát triển bởi Anders Hejlsberg cùng nhóm phát triển .Net Framework của Microsoft và được phê duyệt bởi European Computer Manufacturers Association (ECMA) và International Standards Organization (ISO).

Quá trình dịch chương trình trong C\#  được thiết kế cho các ngôn ngữ chung cơ sở hạ tầng (Common Language Infrastructure – CLI), trong đó bao gồm các mã (Executable Code) và môi trường thực thi (Runtime Environment) cho phép sử dụng các ngôn ngữ cấp cao khác nhau trên đa nền tảng máy tính và kiến trúc khác nhau.

\subsubsection*{Ngôn ngữ ra đời cùng với .NET}
\begin{itemize}
	\item Kết hợp C++ và Java.
	\item Hướng đối tượng.
	\item Hướng thành phần.
	\item Mạnh mẽ (robust) và bền vững (durable).
	\item Mọi thứ trong Quá trình dịch chương trình trong C\#  đều Object oriented.
	\item Chỉ cho phép đơn kế thừa.
	\item Lớp Object là cha của tất cả các lớp.
	\item Cho phép chia chương trình thành các thành phần nhỏ độc lập nhau.
	\item Mỗi lớp gói gọn trong một file, không cần file header như C/C++.
	\item Bổ sung khái niệm namespace để gom nhóm các lớp.
	\item Bổ sung khái niệm “property” cho các lớp.
	\item Khái niệm delegate và event.
\end{itemize}

\subsubsection*{Vai trò C\#  trong .NET Framework}
\begin{itemize}
	\item .NET runtime sẽ phổ biến và được cài trong máy client.
	\begin{enumerate}
		\item Việc cài đặt App C\#  như là tái phân phối các thành phần .NET
		\item Nhiều App thương mại sẽ được cài đặt bằng Quá trình dịch chương trình trong C\# .
	\end{enumerate}
	\item C\# tạo cơ hội cho tổ chức xây dựng các App Client/Server n-tier.
	\item Kết nối ADO.NET cho phép truy cập nhanh chóng và dễ dàng với SQL Server, Oracle…
	\item Cách tổ chức .NET cho phép hạn chế những vấn đề phiên bản.
	\item ASP.NET viết bằng Quá trình dịch chương trình trong C\#.
	\begin{enumerate}
		\item GUI thông minh.
		\item Chạy nhanh hơn (đặc tính của .NET)
		\item Mã ASP.NET không còn hỗn độn.
		\item Khả năng bẫy lỗi tốt, hỗ trợ mạnh trong quá trình xây dựng App Web.
	\end{enumerate}
\end{itemize}

\subsubsection*{Quá trình dịch chương trình trong C\# }
Mã nguồn C\# là các tập tin *.cs được trình biên dịch Compiler biên dịch thành các file *.dll hoặc *.exe, sau đó các file này được các hệ thống thông dịch CLR trên điều hành thông dịch qua mã máy và dùng kỹ thuật JIT (just-in-time) để tăng tốc độ.

\begin{center}
	\begin{figure}[htp]
		\begin{center}
			\includegraphics[scale=.4]{quatrinhthongdich.png}
		\end{center}
		\caption{Quá trình dịch chương trình trong C\#}
		
	\end{figure}
\end{center}

\subsubsection*{Các loại ứng dụng của C\#}
C\# có thể tạo ra được nhiều loại ứng dụng, trong đó có 3 kiểu phổ biến được nhiều nhà lập trình viên sử dụng nhất đó là: Console, Window và ứng dụng Web.
\begin{itemize}
	\item Ứng dụng Console là ứng dụng có giao diện text, chỉ xử lý nhập xuất trên màn hình Console, tương tự với các ứng dụng DOS trước đây. Loại ứng dụng Console thường đơn giản, ta có thể nhanh chóng tạo chương trình hiển thị kết xuất trên màn hình. Do đó, các minh hoạ, ví dụ ngắn gọn ta thường sử dụng dạng chương trình Console để thể hiện.
	\item Ứng dụng Windows Form là ứng dụng được hiển thị với giao diện cửa sổ đồ họa. Chúng ta chỉ cần kéo và thả các điều khiển (control) lên cửa sổ Form. Visual Studio sẽ sinh mã trong chương trình để tạo ra, hiển thị các thành phần trên cửa sổ.
	\item Ứng dụng Web, trên môi trường .NET cung cấp công nghệ ASP.NET, MVC giúp xây dựng những trang Web động. Để tạo ra một trang Web, người lập trình sử dụng ngôn ngữ biên dịch như C\# hoặc C\# để viết mã. Để đơn giản hóa quá trình xây dựng giao diện người dùng cho trang Web, .NET giới thiệu công nghệ Webform. Cách thức tạo ra các Web control tương tự như khi ta xây dựng ứng dụng trên Window Form.
\end{itemize}

\subsection{Công cụ sinh dữ liệu thử Pex}
\subsubsection*{Giới thiệu}
Khái niệm về DSE và các ứng dụng sử dụng kỹ thuật DSE đã có từ lâu, nhưng Pex là một ứng dụng mỡ rộng hơn so với các phiên bản DSE trước. Trong Visual Stuio, Pex đã được tích hợp như một Add-in, và có thể tạo ra các test case kết hợp với các bộ kiểm thử khác nhau như NUnit và MSTest. 

Cũng như với Unit Test, ta có thể viết các lớp kiểm thử chứa các ca kiểm thử tham số hóa. Với sự hỗ trợ của Pex ta có thể thực thi các ca kiểm thử tham số hóa đó. Tuy nhiên không giống việc thực thi các lớp kiểm thử chứa các Unit Test, Pex chỉ thực thi được một ca kiểm thử tham số hóa trong mỗi lần chạy.

\begin{lstlisting}[language={[Sharp]C}, caption={Ca kiểm thử tham số sử dụng Pex}, label={Script}]
[PexMethod]
public void AddSpec(ArrayList list, object element) {
// assumptions
PexAssume.IsTrue(list != null);
// method sequence
int len = list.Count; 
list.Add(element);
// assertions
Assert.IsTrue(list[len] == element);
}
\end{lstlisting}


\subsubsection*{Các mẫu kiểm thử tham số hóa}
Viết các ca kiểm thử các tham số là một công việc tốn nhiều công sức. Để viết các ca kiểm thử các tham số hiệu quả, ta cần thực sự hiểu về mã cài đặt của chương trình mà ta muốn kiểm thử. Pex hỗ trợ cho chúng ta nhiều mẫu kiểm thử tham số khác nhau \cite{de2008parameterized}. Các mẫu được sử dụng nhiều nhất đó là mẫu AAA (Triple-A) và AAAA:
\begin{itemize}
	\item Với mẫu AAA (Arrange, Act, Assert) PUT được tổ chức thành 3 phần:
	\begin{itemize}
		\item Arrange: khởi tạo giá trị các biến sẽ sử dụng
		\item Act: dãy các lời gọi phương thức
		\item Assert: sự xác nhận
	\end{itemize}
	\item Với mẫu AAAA, một giả thuyết (Assume) được thêm vào để giới hạn miền giá trị của các tham số đầu vào.
\end{itemize}

\begin{lstlisting}[language={[Sharp]C}, caption={Mầu kiềm thử tham số hóa AAAA}, label={Script}]
[PexMethod]
void AssumeActAssert(ArrayList list, object item) { 
// assume
PexAssume.IsNotNull(list);
// arrange
var count = list.Count;
// act
list.Add(item);
// assert
Assert.IsTrue(list.Count == count + 1);
}
\end{lstlisting}

\subsubsection*{Lọn chọn đầu vào kiểm thử với Pex}
Đổ có thể sinh các đầu vào cụ thể cho các tham số Unit test, Pex cần phải phân tích chương trình với các tham số kiểm thử này. Có 2 kỹ thuật phân tích chương trình đó là:

\begin{itemize}
	\item Phân tích tĩnh (static analysis): Kiểm chứng một tính chất nào đó của chương trình bằng việc phân tích tất cả các đường đi thực thi. Kỹ thuật này coi các cảnh bảo (violations) là các lỗi (error).
	\item Phân tích động (dynamic analysis): Kiểm chứng một tính chất bằng việc phân tích một số đường đi thực thi. Đây là một kỹ thuật phân tích động hỗ trợ việc phát hiện ra các lỗi (bugs) nhưng không khẳng định được rằng có còn những lỗi khác hay không. Các kỹ thuật này thường không tìm ra được tất cả các lỗi.
\end{itemize}

Pex cài đặt một kỹ thuật phân tích chương trình bằng cách kết họp cả hai kỹ thuật phân tích chương trình ở trên gọi là thực thi tượng trưng động \cite{xie2009fitness}, \cite{godefroid2005dart}. Về bản chất Pex là một công cụ hỗ trợ kỹ thuật kiểm thử hộp hắng (white-box testing). Tương tự như kỹ thuật phân tích chương trình tĩnh, Pex chứng minh được rằng một tính chất được kiểm chứng trong tất cả các đường đi khả thi. Pex chỉ báo cáo (reporting) về các lỗi thực sự như với kỹ thuật phân tích chương trình động.

Pex sử dụng bộ xử lý ràng buộc Z3 \cite{de2008z3} kết họp với các lý thuyết toán học khác như hàm chưa định nghĩa, lý thuyết mảng, bit-vetor \cite{kroening2016decision} để giải quyết ràng buộc sinh ra trong quá trình thực thi tượng trưng động và sinh ra các đầu vào kiểm thử cụ thể cho tham số kiểm thử.

\subsubsection*{Mô hình ứng dụng Pex}

\begin{center}
	\begin{figure}[htp]
		\begin{center}
			\includegraphics[scale=.4]{pex.png}
		\end{center}
		\caption{Mô hình ứng dụng Pex}
		
	\end{figure}
\end{center}

\section{Đánh giá kết quả thực nghiệm}
Phần này trình bày những kết quả đo được trên bộ dữ liệu thực nghiệm đã nêu trong Phần~\ref{sec:data}.

\section{Khả năng ứng dụng}
Phần này trình bày một số ứng dụng có thể của việc đo độ tương tự về hành vi của chương trình cùng hướng phát triển trong tương lai.

\subsection{Đánh giá tiến bộ trong lập trình}
Theo dõi sự tiến bộ trong học tập là một việc quan trọng, mà ngay cả với giảng viên và sinh viên công việc này là cả một quá trình. Có nhiều tiêu chí đánh giá sự tiến bộ trong học tập của sinh viên, trong đó tiêu chí về điểm số là một trong những tiêu chí cơ bản nhất. Một bảng thống kê điểm số, thành tích học tập của sinh viên sẽ thể hiện được sự tiến bộ của sinh viên trong học tập. Nhưng công việc chấm bài, tổng hợp, thống kê kết quả học tập của sinh viên mất rất nhiều thời gian của giảng viên. Một ứng dụng hỗ trợ chấm điểm, lưu trữ, thống kê và đánh giá điểm số của sinh viên là sẽ là một công cụ hỗ trợ đắc lực cho giảng viên trong công tác quản lý của mình. Nếu số liệu thống kê đánh giá kết quả các bài kiểm tra của sinh viên ngày càng cao, chứng tỏ sinh viên nắm được nội dung và kiến thức của chương trình đào tạo, và kết quả tốt sẽ là một động lực giúp cho sinh viên thêm tự tin, đam mê công việc học tập của mình. Ngược lại, nếu một sinh viên có điểm số ngày càng thấp chứng tỏ sinh viên đang có vấn đề trong kiến thức của của mình, lúc này tốt nhất sinh viên nên dừng lại không tiếp tục code và kiểm tra xem vấn đề mình đang gặp phải. 

\subsection{Xếp hạng tự động}
Công việc chấm điểm, phân loại và xếp hạng các bài kiểm tra của sinh viên cũng là một công việc tốn không ít công sức của giảng viên. Để giảm bớt gánh nặng cho giảng viên, chúng ta có thể sử dụng kết quả các độ đo trên từng bài tập của sinh viên như một phương pháp hỗ trợ công việc chấm điểm của từng sinh viên. Sự giống nhau về hành vi giữa chương trình của sinh viên và chương trình tham chiếu có thể là một yếu tố để phân loại sinh viên. Độ tương tự càng cao thì điểm số càng cao, các chỉ số này dựa hoàn toàn trên ngữ nghĩa của chương trình. Cách tiếp cận này giải quyết được các giới hạn trong trường hợp chương trình của sinh viên giống với chương trình tham chiếu, nhưng khác nhau về ngữ nghĩa. Các kết quả trong việc xếp hạng tự động sẽ giúp tiết kiệm được thời gian và giảng viên có thể đưa ra giải pháp giúp những sinh viên có điểm số thấp khắc phục được hạn chế đang gặp phải.

\subsection{Gợi ý giải pháp lập trình}
Thông thường, sinh viên thường viết code mới thực hiện chạy chương trình, lúc này sinh viên mới biết được kết quả đoạn code vừa thực hiện. Để hỗ trợ sinh viên viết code được tốt hơn, nếu như có một công cụ hỗ trợ kiểm tra theo thời gian thực và gửi thông báo lỗi nếu sinh viên viết code sai cú pháp hoặc chương trình bị lỗi không thể thực thi được. Ngoài ra, công cụ sẽ gợi ý giải pháp lập trình cho sinh viên bằng hình thức tự động tính toán thông báo kết quả các tham số đầu vào và đầu ra của chương trình so với chương trình được tham chiếu, đưa ra các số liệu về độ tương tự hành vi của chương trình.	

\subsection{Hướng phát triển}
Qua quá trình nghiên cứu và triển khai thực nghiệm, trong tương lai đề tài hướng tới phát triển thành một một ứng dụng hoàn chỉnh với việc bổ sung và hoàn thiện một số chức năng như sau:
\begin{itemize}
	\item Phát triển ứng dụng có thể chạy trên Web
	\item Chức năng quản lý sinh viên
	\item Quản lý kết quả học tập sinh viên
	\item Thêm chức năng đánh giá, xếp hạng tự động
	\item Thêm chức năng gợi ý giải pháp lập trình
	\item Cải tiến các độ đo để cho kết quả tốt hơn và nhanh hơn
	\item Phát triển thêm các nền tảng lập trình khác như java, c++..		
\end{itemize}

\section{Kết luận}
Qua quá trình nghiên cứu đề tài, có thể thấy rằng việc phát triển và ứng dụng các kỹ thuật đo đang dần trở nên phổ biến trong các chương trình giáo dục và đào tạo lập trình viên online. Những lợi ích, hiểu quả của việc đánh giá độ tương tự hành vi mang lại là rất thiết thực. Các kỹ thuật này không chỉ giúp quá trình giảng dạy của giảng viên được thuận lợi hơn, tiết kiệm được thời gian cũng như công sức trong công tác quản lý. Ngoài ra, sinh viên có được một môi trường tốt để tự rèn luyện, nâng cao các kỹ năng lập trình của bản thân. Việc tạo động lực giúp sinh viên có sự hứng thú và đam mê lập trình là rất cần thiết. Một khi sinh viên có tư duy và kỹ năng lập trình tốt, sinh viên sẽ tự tin vào năng lực của bản thân để tiếp tục phát phát triển sự nghiệp sau khi ra trường.

Đề tài đã thực hiện nghiên cứu rất nhiều vấn đề, nhưng trọng tâm đó là nghiên cứu sơ lượt về công cụ PEX của Microsoft, một công cụ sử dụng bộ xử lý ràng buộc Z3 \cite{de2008z3} kết họp với các lý thuyết toán học khác như hàm chưa định nghĩa, lý thuyết mảng, bit-vetor \cite{kroening2016decision} để giải quyết các ràng buộc sinh ra trong quá trình thực thi tượng trưng động, và sinh ra các đầu vào kiểm thử có độ phủ cao cho tham số kiểm thử của chương trình. Ứng dụng công cụ PEX nghiên cứu các kỹ thuật đo độ tương tự hành vi của chương trình, sao cho kết quả các độ đo được chính xác hơn. Từ khả năng xử lý của các độ đo, xây dựng một công cụ để minh họa cho các kỹ thuật đo.

Những nội dung trình bày trong luận văn này không tránh khỏi còn nhiều thiếu xót vì lý do nhiều lý do khách quan khác nhau như: Thời gian thực hiện đề tài hạn hẹp; lượng kiến thức cơ sở để triển khai thực hiện rất lớn; cùng với đó là kinh nghiệm của bản thân trong việc thực hiện các đề tài chưa có. Tuy nhiên, với sự hướng dẫn tận tình của giáo viên hướng dẫn, đề tài đã đạt được nhiều nội dung quan trọng, và có thể mở ra nhiều hướng nghiên cứu và phát triển khác của đề tài. Như việc mở rộng DSE để hỗ trợ việc thực thi tượng trưng các chương trình có dữ liệu phức tạp, hay có số vòng lặp lớn. Cải tiến kỹ thuật đo, kết hợp với kỹ thuật DSE để có kết quả độ đo được chính xác hơn, nhạy hơn. Tiếp tục nghiên cứu, phát triển để thực các kỹ thuật đo trên các ngôn ngữ khác như Java, C++ và chạy được trên nền tảng PC, Web. 



\bibliographystyle{plain}
\bibliography{biblio}
\todo{Trích dẫn bài báo}
\nocite{*} % dùng tạm thời


\appendix

\chapter{QUYẾT ĐỊNH GIAO LUẬN VĂN}
\newpage
\chapter*{QUYẾT ĐỊNH GIAO LUẬN VĂN} 
\addcontentsline{toc}{chapter}{QUYẾT ĐỊNH GIAO LUẬN VĂN}
\todo{scan quyet dinh sang pdf và include}
%\includepdf{quyetdinh.pdf}

\chapter{Phụ lục XXX}
\begin{center}
	\textbf{PHỤ LỤC}
\end{center}

\chapter{MỘT SỐ MÃ LỆNH QUAN TRỌNG}
\lstinputlisting[caption = {Mã lệnh tạo project của sinh viên}]{MakeProjects.cs}

\lstinputlisting[caption= {Mã lệnh tạo project chương trình tham chiếu}]{MakeSecretProjects.cs}

\lstinputlisting[caption={Mã lệnh build project của sinh viên}]{BuildProjects.cs}

\lstinputlisting[caption={Mã lệnh build project chương trình tham chiếu}]{BuildSecretProjects.cs}

\lstinputlisting[caption={Mã lệnh build project chương trình hợp thành}]{BuildMetaProjects.cs}

\lstinputlisting[caption={Mã lệnh thực thi DSE trên chương trình hợp thành}]{RunPexOnMetaProjects.cs}

\lstinputlisting[caption={Mã lệnh phép đo RS}]{ComputeMetric3.cs}

\lstinputlisting[caption={Mã lệnh phép đo SSE}]{ComputeMetric2.cs}

\lstinputlisting[caption={Mã lệnh phép đo PSE}]{ComputeMetric1.cs}

 


\end{document}

