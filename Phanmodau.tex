\section{Lý do chọn đề tài}
Hiện nay, ngành Công nghệ thông tin đang có xu hướng phát triển mạnh mẽ và trở nên phổ biến. Một số chương trình đào tạo nổi tiếng như
\href{https://www.coursera.org/course/saas}{Massive Open Online
  Courses (MOOC)} \cite{mooc}, \href{https://www.edx.org/}{edX}
\cite{edx}, \href{https://www.coursera.org/}{Coursera}
\cite{coursera}, \href{http://www.udacity.com/}{Udacity}
\cite{Udacity}, chương trình
học lập trình online như \href{https://www.pexforfun.com/}{Pex4Fun}
\cite{Pex4Fun} hay
\href{https://www.microsoft.com/en-us/research/project/code-hunt/}{Code
  Hunt} \cite{CodeHunt} thu hút nhiều sự quan tâm của các bậc phụ huynh và các em học sinh.

Những lớp học như vậy thường có nhiều sinh viên tham gia, các em sinh viên đến từ nhiều nơi trên thế giới với nhiều nền văn hóa và ngôn ngữ khác nhau. Để có thể quản lý được những lớp học như vậy là cả một khối lượng công việc lớn, yêu cầu nỗ lực của những người quản lý và giảng viên. Chỉ riêng việc đọc hiểu để đánh giá kết quả mã lệnh do sinh viên viết đã tốn nhiều thời gian, nếu như bỏ qua thì giảng viên sẽ không theo dõi được quá trình học tập của sinh viên. Những lúc khó khăn trong việc viết mã chương trình, sinh viên
có thể nhờ bạn bè hoặc nhờ những người có kinh nghiệm giúp đỡ. Tuy nhiên, không phải lúc nào cũng có người bên cạnh để giúp đỡ cho họ, và kinh nghiệm cũng như kiến thức của những người này chưa chắc có thể đáp ứng được yêu cầu.

Để giảm bớt khó khăn cho giảng viên và sinh viên, một công cụ hỗ trợ tự động đánh giá kết quả chương trình của sinh viên với chương trình của giảng viên sẽ giúp tiết kiệm được thời gian, giúp cho giảng viên quản lý việc học tập của sinh viên được tốt hơn. Sinh viên có thể nhanh chóng biết được chương trình của mình viết đúng hay sai.

Cách thức hoạt động của công cụ này là đánh giá độ tương tự về hành vi của hai chương trình. Công cụ sẽ tính toán để tìm ra các mẫu dữ liệu đầu vào chung cho cả hai chương trình, tiến hành lấy từng mẫu dữ liệu đầu vào chạy đồng thời trên cả hai chương trình và so sánh kết quả đầu ra của hai chương trình. Nếu kết quả đầu ra của hai chương trình có tỷ lệ giống nhau càng cao thì điểm số của sinh viên càng cao. Ngược lại, nếu tỷ lệ càng thấp thì tương ứng với điểm số của sinh viên càng thấp. Dựa trên kết quả này, giảng viên có thể nắm bắt được tình hình học tập của sinh viên và có hướng khắc phục những hạn chế mà sinh viên đang gặp phải. Đây cũng là một cách giúp cho sinh viên không đi lệch khỏi định hướng kiến thức, các kỹ thuật, kỹ năng lập trình và hạn chế được những nguy cơ tìm ẩn trong cách viết mã lệnh chương trình. Đồng thời giúp tiết kiệm được thời gian cho cả giảng viên và sinh viên.

\section{Đối tượng, phạm vi, phương pháp nghiên cứu}
\subsection*{Mục tiêu nghiên cứu}

Mục tiêu nghiên cứu chính của luận văn là đánh giá độ tương tự về hành vi của các chương trình.
		
\subsubsection*{Mục tiêu nghiên cứu cụ thể}
\begin{itemize}
\item Tìm hiểu sự tương tự hành vi của chương trình
\item Tìm hiểu kỹ thuật, công cụ sinh Test Case tự động và áp dụng kỹ thuật sinh Test Case tự động trên các kỹ thuật đo độ tương tự
\item Tìm cách kết hợp các kỹ thuật đo với nhau
\item Đánh giá kết quả thực nghiệm
\end{itemize}

\subsection*{Đối tượng, phạm vi nghiên cứu}	
\subsubsection*{Đối tượng nghiên cứu}
\begin{itemize}
\item Kỹ thuật sinh Test Case
\item Các kỹ thuật đo độ tương tự hành vi
\item Ứng dụng của các kỹ thuật đo độ tương tự hành vi
\end{itemize}
	
\subsubsection*{Phạm vi nghiên cứu}
\begin{itemize}
\item Đo độ tương tự hành vi dựa vào Test Case
\item Thực nghiệm, đánh giá trên các chương trình C\#
\end{itemize}


\subsection*{Phương pháp nghiên cứu, thực nghiệm}
\subsubsection*{Nghiên cứu lý thuyết}
\begin{itemize}
\item Độ tương tự hành vi
\item Một số kỹ thuật sinh Test Case tự động
\item Kỹ thuật đo độ tương tự hành vi dựa trên Test Case
\item So sánh, kết hợp các phép đo độ tương tự hành vi
\end{itemize}
		
\subsubsection*{Thực nghiệm}
\begin{itemize}
\item Tiến hành cài đặt các kỹ thuật đo độ tương tự hành vi
\item Thực nghiệm trên dữ liệu thực của CodeHunt, và một số dữ liệu thử khác
\item Phân tích, đánh giá dựa trên kết quả thực nghiệm
\end{itemize}





