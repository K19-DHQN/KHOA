Chương này trình bày một số định nghĩa liên quan đến hành vi chương trình, các phép đo độ tương tự hành vi của chương trình và tiêu chí đánh giá các phép đo này.

\section{Hành vi của chương trình }
Để định lượng hai chương trình tương tự nhau, chúng ta nghiên cứu các định nghĩa liên quan hành vi của hai chương trình như sau:
	
\subsection{Thực thi chương trình}
\begin{definition}\label{def:progexe}
Cho $P$ là một chương trình, $I$ là tập hợp các trị đầu vào của $P$ và $O$ là tập hợp các giá trị đầu ra của $P$. Thực thi chương trình P là ánh xạ $exec: P \times I \rightarrow O$. Với giá trị đầu vào $i \in I$, sau khi thực thi $P$ trên $i$ ta có giá trị đầu ra tương ứng $o \in O$ và ký hiệu $o = exec(P, i)$.  
\end{definition}

\subsection{Độ tương đương về hành vi (Behavioral Equivalence)}

Dựa trên định nghĩa về thực thi chương trình, chúng ta tìm hiểu thế nào là độ tương đương về hành vi giữa hai chương trình thông qua hai chương trình minh họa sau:

\begin{minipage}[t]{0.45\linewidth}
	\lstinputlisting[caption = {Switch...Case}]{SwitchCase.cs}
\end{minipage}%
\hfill\vrule\hfill
\begin{minipage}[t]{0.45\linewidth}
	\lstinputlisting[caption = {If...Else}]{IfElse.cs}
\end{minipage}%

Mã lệnh $3.1$ và $3.2$ có tham số đầu vào cùng kiểu giá trị $int$. Mã lệnh $3.1$ sử dụng cấu trúc $\textit{\textbf{switch...case}}$, mã lệnh $3.2$ sử dụng cấu trúc $\textit{\textbf{if...else}}$ để kiểm tra giá trị đầu vào $x$. Mặc dù cú pháp sử dụng trong hai chương trình là khác nhau nhưng cách thức xử lý trả về kết quả $y$ là như nhau. Từ đó, chúng ta có thể định nghĩa thế nào là độ tương đương hành vi giữa hai chương trình như sau:

\begin{definition}[Độ tương đương về hành vi]
 Cho $P_{1}$ và $P_{2}$ là hai chương trình có cùng miền các giá trị đầu vào $I$. Hai chương trình này được gọi là tương đương khi và chỉ khi thực thi của chúng giống nhau trên mọi giá trị đầu vào trên $I$, ký hiệu là exec($P_{1}, I$) = exec($P_{2}, I$). 
\end{definition}	
	
\subsection{Sự khác biệt về hành vi (Behavioral Difference) }
Để tìm hiểu sự khác biệt về hành vi của hai chương trình, chúng ta tìm hiểu hai mã lệnh sau:

\begin{minipage}[t]{0.45\linewidth}
	\lstinputlisting[caption = {Chương trình P\_1}]{Khac_biet_HV_1.cs}
\end{minipage}%
\hfill\vrule\hfill
\begin{minipage}[t]{0.45\linewidth}
	\lstinputlisting[caption = {Chương trình P\_2}]{Khac_biet_HV_2.cs}
\end{minipage}%

Mã lệnh $3.3$ và Mã lệnh $3.4$ của Chương trình $P_{1}$ và $P_{2}$, cả hai Chương trình có miền giá trị đầu vào cùng kiểu $int$, giá trị trả về của Chương trình $P_{1}$ là $x - 10$, giá trị trả về của Chương tình $P_{2}$ là $x + 10$. Với mọi giá trị của $x$ được thực thi trên cả hai chương trình $P_{1}$ và $P_{2}$ kết quả trả về sẽ không giống nhau. Mặc dù cả hai Chương trình $P_{1}$ và $P_{2}$ có miền giá trị đầu vào như nhau, nhưng hành vi của hai Chương trình hoàn toàn khác nhau. Qua đó, chúng ta có thể định nghĩa sự khác biệt về hành vi như sau:

\begin{definition}[Sự khác biệt hành vi]
Cho $P_{1}$ và $P_{2}$ là hai chương trình có cùng một miền các giá trị đầu vào $I$. Hai chương trình này được xem là có sự khác biêt về hành vi khi và chỉ khi thực thi của chúng khác nhau trên mọi giá trị đầu vào $I$, ký hiệu là $exec(P_{1}, I) \neq exec(P_{2}, i)$.
\end{definition}

\subsection{Độ tương tự hành vi (Behavioral Similarity)}
Để hiểu thế nào là tương tự hành vi, chúng ta phân tích hai Mã lệnh sau:

\begin{minipage}[t]{0.45\linewidth}
	\lstinputlisting[caption = {Chương trình $P_{1}$}]{TuongTu_HV_1.cs}
\end{minipage}%
\hfill\vrule\hfill
\begin{minipage}[t]{0.45\linewidth}
	\lstinputlisting[caption = {Chương trình $P_{2}$}]{TuongTu_HV_2.cs}
\end{minipage}%

Với Mã lệnh $3.5$ và Mã lệnh $3.6$ của hai Chương trình $P_{1}$ và $P_{2}$, chúng ta thấy cả hai Chương trình có giá trị đầu vào cùng kiểu dữ liệu là $int$, nếu giá trị đầu vào của biến $x$ nằm trong khoảng $0$ đến $100$ thì giá trị trả về của cả hai Chương trình đều bằng nhau là $x+10$. Ngươc lại, giá trị đầu vào của biến $x$ nằm ngoài khoảng $0$ đến $100$, thì giá trị trả về của Chương trình $P_{1}$ là $x$ và giá trị trả về của Chương trình $P_{2}$ là $-1$. Hai Chương trình $P_{1}$ và $P_{2}$ tuy có kiểu dữ liệu đầu vào như nhau nhưng kết quả đầu ra có thể giống nhau hoặc khác nhau tùy theo giá trị đầu vào của biến $x$. Dựa trên kết quả phân tích, chúng ta định nghĩa độ tương tự hành vi của Chương trình như sau:

\begin{definition}[Độ tương tự hành vi]
Cho $P_{1}$ và $P_{2}$ là hai chương trình có cùng miền giá trị đầu vào $I$, và $I_{s}$ là tập con của $I$. Hai Chương trình được xem là tương tự hành vi khi thực thi chúng giống nhau trên mọi giá trị đầu vào $I_{s}$, ký hiệu  $exec(P_{1}, I_{s}) = exec(P_{2}, I_{s})$ và khác nhau $\forall j \in I \setminus I_{s}$, ký hiệu $exec(P_{1}, j) \neq exec(P_{2}, j)$
\end{definition}

\section{Một số phép đo độ tương tự hành vi}
Để đo độ tương tự về hành vi giữa hai Chương trình, chúng ta có thể chạy từng giá trị đầu vào trong miền giá trị đầu vào của hai Chương trình. Tỷ lệ giữa số lượng đầu vào thử nghiệm khi thực thi trên cả hai Chương trình cho kết quả đầu ra giống nhau trên tổng số lượng đầu vào được thử nghiệm là độ tương tự hành vi giữa hai Chương trình. Dựa trên cách tính tỷ lệ kết quả đầu ra của hai Chương trình, chúng ta có một số phép đo độ tương tự hành vi như sau:

\subsection{Phép đo lấy mẫu ngẫu nhiên (RS)}
Kỹ thuật của phép đo \textbf{RS} là thực hiện lấy ngẫu nhiên giá trị đầu vào trên miền giá trị đầu vào của hai Chương trình. Thực thi cả hai Chương trình trên từng giá trị đầu vào, tiến hành so sánh giá trị kết quả đầu ra của cả hai Chương trình. Tỷ lệ giữa tổng số mẫu đầu vào khi thực thi những mẫu này hai chương trình cho kết quả đầu ra có giá trị giống nhau, trên tổng số mẫu đầu vào được thử nghiệm là kết quả cho phép đo RS. Từ đó, chúng ta có định nghĩa phép đo \textbf{RS} như sau:
 
\begin{definition}[Phép đo RS]
Cho $P_{1}$ và $P_{2}$ là hai Chương trình có cùng miền giá trị đầu vào $I$, $I_{s}$ là tập con ngẫu nhiên của $I$, $I_{a}$ là tập con $I_{s}$. Hai Chương trình thực thi giống nhau với $\forall i \in I_{a}$, ký hiệu $exec(P_{1}, i) = exec(P_{2}, i)$ và hai Chương trình thực thi khác nhau với $\forall j \in I_{s} \setminus I_{a}$, ký hiệu $exec(P_{1}, j) \neq exec(P_{2}, j)$. Chỉ số phép đo \textbf{RS} được định nghĩa là $M_{RS}(P_{1}, P_{2}) = \left|I_{a}\right| \diagup \left|I_{s}\right| $.
\end{definition}

Phép đo \textbf{RS} là một một phép đo đơn giản và hiệu quả để tính độ tương tự của hành vi. Khi miền giá trị của tham số đầu vào rất lớn hoặc vô hạn, phép đo \textbf{RS} thực hiện lấy mẫu ngẫu nhiên để tính toán độ tương tự của hành vi, kết quả đầu ra tương đối tốt và hợp lý so với hành vi thực tế của chương trình. Phép đo \textbf{RS} xử lý, tính toán độ tương tự hành vi dưới dạng hộp đen và không phân tích chương trình để tạo thử nghiệm, nên tốc độ xử lý nhanh và chiếm ít tài nguyên. Mặc khác, phép đo \textbf{RS} không phân tích chương trình để tạo đầu vào thử nghiệm nên phép đo \textbf{RS} có thể bỏ qua một vài tham số đầu vào thử nghiệm có thể được sử dụng để thực thi một số nhánh khác nhau giữa hai chương trình. Vì vậy, phép đo \textbf{RS} không phân biệt được các chương trình có một số hành vi khác nhau. Chúng ta phân tích Mã lệnh $3.7$ và $3.8$ sau để thấy được hạn chế của phép đo \textbf{RS}:

\begin{minipage}[t]{0.45\linewidth}
	\begin{lstlisting}[caption={Chương trình $P_{1}$}, label={Script}]
	public static int Y(string x) {
		if (x == "XYZ") return 0;
		if (x == "ABC")	return 1	
		return -1;
	}
	\end{lstlisting}
\end{minipage}%
\hfill\vrule\hfill
\begin{minipage}[t]{0.45\linewidth}
	\begin{lstlisting}[caption={Chương trình $P_{2}$}, label={Script}]
	public static int Y(string x) {
		if (x == "ABC")	return 1	
		return -1;
	}
	\end{lstlisting}
\end{minipage}%

Chúng ta thấy đoạn Mã lệnh $3.7$ và $3.8$ của hai Chương trình $P_{1}$ và $P_{2}$ có cùng miền giá trị đầu vào là $string x$, cấu trúc mã lệnh hai chương trình gần như nhau. Nhưng Chương trình $P_{1}$ khác với Chương trình $P_{2}$ đó là sẽ trả kết quả về $0$ nếu tham số đầu vào có giá trị là $XYZ$. Tỷ lệ phép đo \textbf{RS} lấy ngẫu nhiên giá trị đầu vào $x$ trên miền giá trị đầu vào của hai chương trình có giá trị bằng $XYZ$ là rất thấp, vì vậy khả năng câu lệnh $if (x == "XYZ") return 0;$ của Chương trình $P_{1}$ có thể sẽ không được thực thi nên kết quả của phép đo \textbf{RS} sẽ ở mức tương đối so với hành vi thực tế của chương trình.
	
\subsection{Phép đo tượng trưng trên một chương trình Single Program Symbol Execution (SSE)}
Phép đo \textbf{SSE} là một phép đo dựa trên số lượng các nhánh đường đi của Chương trình mẫu, mỗi nhánh đường đi của Chương trình mẫu được xem là một hành vi của chương trình. Nếu chọn một giá trị đầu vào thử nghiệm cho một nhánh đường đi trong Chương trình thì các giá trị đầu vào thử nghiệm này sẽ khám phá hết các hành vi trong Chương trình mẫu. Do vậy, số phần tử trong tập các giá trị đầu vào thử nghiệm của phép đo \textbf{SSE} sẽ nhỏ hơn tập các giá trị đầu vào thử nghiệm được chọn theo phương pháp lấy ngẫu nhiên giá trị đầu vào. 

Để tính độ tương tự hành vi của hai chương trình với phép đo \textbf{SSE}, chúng ta chọn Chương trình mẫu làm Chương trình tham chiếu và áp dụng kỹ thuật \textbf{DSE} để tạo ra các đầu vào thử nghiệm dựa trên Chương trình tham chiếu. Sau đó thực thi cả hai chương trình dựa trên các giá trị đầu vào thử nghiệm. Tỷ lệ số lượng các kết quả đầu ra giống nhau của cả hai chương trình trên tổng số các giá trị đầu vào thử nghiệm của Chương trình tham chiếu là kết quả của phép đo \textbf{SSE}. Qua đó, chúng ta có định nghĩa phép đo \textbf{SSE} như sau:

\begin{definition}
 Cho $P_{1}$ và $P_{2}$ là hai chương trình có cùng miền giá trị đầu vào $I$, Chương trình $P_{1}$ là Chương trình tham chiếu, $I_{s}$ là tập các giá trị đầu vào được tạo bởi DSE trên chương trình $P_{1}$, và $I_{a}$ là tập con $I_{s}$. Hai Chương trình thực thi giống nhau với $\forall i \in I_{a}$, ký hiệu $exec(P_{1}, i) = exec(P_{2}, i)$ và hai Chương trình thực thi khác nhau với $\forall j \in I_{s} \setminus I_{a}$, ký hiệu $exec(P_{1}, j) \neq exec(P_{2}, j)$. Chỉ số phép đo \textbf{SSE} được định nghĩa là $M_{SSE}(P_{1}, P_{2}) = \left|I_{a}\right| \diagup \left|I_{s}\right| $.
\end{definition}

Ngược lại với phép đo RS, phép đo SSE khám phá những đường đi khả thi khác nhau trong chương chình tham chiếu để tạo dữ liệu đầu vào của chương trình. Do đó, các đầu vào thử nghiệm này sẽ thực thi hết các đường đi của chương trình tham chiếu và có khả năng phát hiện được những chương trình cần tính có những hành vi khác so với Chương trình tham chiếu. Những phép đo SSE vẫn còn hạn chế, đó là phép đo SSE không xem xét đường đi của Chương trình cần phân tích để tạo các giá trị đầu vào thử nghiệm mà chỉ dựa vào các đầu vào thử nghiệm được phân tích từ Chương trình tham chiếu. Các đầu vào thử nghiệm này không nắm bắt được hết các hành vi của Chương trình cần phân tích, Chương trình cần phân tích có thể sẽ có những hành vi khác so với Chương trình tham chiếu. Một số chương trình có thể có những vòng lập vô hạn phụ thuộc vào giá trị đầu vào nên SSE không thể liệt kê được tất cả các đường dẫn của chương trình. Chúng ta xem xét và phân tích 2 đoạn Mã lệnh $3.9$ và $3.10$ để thấy được hạn chế của phép đo SSE như sau:

\begin{minipage}[t]{0.45\linewidth}
	\begin{lstlisting}[caption={Chương trình $P_{1}$}, label={Script}]
	public static int sol(int x) {
            int y = 0;
            switch (x)
            {
                case 1: 
                			y += 4;
                			break;                
                default: 
                			y = x - 100; 
                			break;
            }
            return y;        
    }
	\end{lstlisting}
\end{minipage}%
\hfill\vrule\hfill
\begin{minipage}[t]{0.45\linewidth}
	\begin{lstlisting}[caption={Chương trình $P_{2}$}, label={Script}]
	public static int sub(int x) {
            int y = 0;
            if (x == 1) 
            	return y += 4;
            if (x == 2) 
            	return y *= 4;                      
            else 
            	return y = x - 100;
        }
	\end{lstlisting}
\end{minipage}%

Hai đoạn Mã lệnh $3.9$ và Mã lệnh $3.10$ của hai Chương trình $P_{1}$ và $P_{2}$, chọn Chương trình $P_{1}$ làm Chương trình tham chiếu, sử dụng kỹ thuật \textbf{DSE} để phân tích chương trình $P_{1}$ ta được tập các giá trị đầu vào thử nghiệm là ${(0, 1)}$. Trong khi đó, phân tích Chương trình $P_{2}$ chúng ta được tập các giá trị đầu vào thử nghiệm của Chương trình $P_{2}$  là ${(0, 1, 2)}$. Do đó, chúng ta thấy tập giá trị đầu vào thử nghiệm do phép đo \textbf{SSE} tạo ra thiếu giá trị đầu thử nghiệm $2$ để có thể thực thi hết các đường đi của Chương trình $P_{2}$.

\subsection{Kỹ thuật thực thi chương trình kết hợp Paired Program Symbolic Execution (PSE)}
Để giải quyết giới hạn của phép đo \textbf{SSE} khi tạo ra tập các giá trị đầu vào thử nghiệm không thực thi hết các các đi của Chương trình cần phân tích. Phép đo \textbf{PSE} giải quyết giới hạn của phép đo \textbf{SSE} bằng cách tạo một Chương trình kết hợp giữa Chương trình cần phân tích với Chương trình tham chiếu. Dựa trên Chương trình kết hợp sử dụng kỹ thuật \textbf{DSE} để tạo ra đầu vào thử nghiệm cho cả hai chương trình, các đầu vào thử nghiệm này bao gồm các đầu vào thử nghiệm đúng và không đúng. Các đầu vào thử nghiệm đúng là những giá trị khi thực thi trên cả hai chương trình sẽ cho kết quả đầu ra như nhau, ngược lại các đầu vào thử nghiệm không đúng là những giá trị khi thực thi trên cả hai chương trình sẽ cho kết quả khác nhau. Do đó, phép đo \textbf{PSE} được tính bằng tỷ lệ các giá trị đầu vào thử nghiệm đúng trên tổng số các giá trị đầu vào được thử nghiệm. 
\begin{lstlisting}[caption={Chương trình $P_{3}$, label={Script}]
	public int $P_{3}$ (int number) {
		if(Program1(args) == Program2(args))
			return 1;
		return 0;
	}
\end{lstlisting}
Dựa trên cách thức hoạt động của phép đo \textbf{PSE}, chúng ta có định nghĩa phép đo \textbf{PSE} như sau:
\begin{definition}[Phép đo PSE]
Cho $P_{1}$ và $P_{2}$ là hai Chương trình có cùng miền giá trị đầu vào $I$. $P_{3}$ là Chương trình
kết hợp của $P_{1}$ và $P_{2}$, ký hiệu $(exec(P_{1}, I) =  exec(P_{2}, I))$.  $I_{s}$ là tập các giá trị đầu vào được tạo bởi \textbf{DSE} từ Chương trình $P_{3}$, $I_{a}$ là tập con $I_{s}$. Hai Chương trình $P_{1}$ và $P_{2}$ thực thi giống nhau khi Chương trình $P_{3}$ thực thi cho giá trị đúng với $\forall i \in I_{a}$, ký hiệu $exec(P_{3}, i) = T$, và $\nexists j \in I_{s} \backslash I_{a}$ thực thi Chương trình $P_{3}$ cho giá trị đúng. Chỉ số phép đo \textbf{PSE} được định nghĩa là $M_{PSE}(P_{1}, P_{2}) = \left|I_{a}\right| \diagup \left|I_{s}\right| $.
\end{definition}
Phép đo \textbf{PSE} đã cải thiện được hạn chế của phép đo \textbf{SSE} khi dữ liệu thử nghiệm được  tạo ra từ trên Chương trình kết hợp, tập dữ liệu thử nghiệm có khả năng thực thi hết các nhánh đường đi của Chương trình tham chiếu và Chương trình cần tính. Tuy nhiên, phép đo \textbf{PSE} cũng có hạn chế trong quá trình xử lý các vòng lặp lớn hoặc vô hạn. Để giảm bớt hạn chế này, chúng ta có thể giới hạn miền đầu vào hoặc đếm số vòng lặp của các Chương trình. Ngoài ra, phép đo \textbf{PSE} khám phá đường dẫn của Chương trình kết hợp nên quá trình xử lý sẽ tốn thời gian và tài nguyên hơn so với phép đo \textbf{SSE} khi chỉ khám phá đường dẫn của Chương trình tham chiếu.
\section{Tiêu chí đánh giá hiệu quả}
Để đánh giá độ hiệu quả của các phép đo, chúng ta có thể áp dụng những tiêu chí cơ bản sau:
\begin{itemize}
\item Tốc độ xử lý
\item Sử dụng tài nguyên
\item Độ phủ của dữ liệu thử
\item Kết quả đánh giá độ tương tự hành vi
\end{itemize}
Phép đo \textbf{RS} sử dụng kỹ thuật lấy ngẫu nhiên giá trị thử nghiệm trong miền giá trị đầu vào của cả hai Chương trình nên tốc độ xử lý của phép đo \textbf{RS} nhanh, đơn giản và sử dụng ít tài nguyên. Nhưng độ phủ dữ liệu thử nghiệm do phép đo \textbf{RS} tạo ra  không cao, không phủ hết các trường hợp có thể thực thi của chương trình nên kết quả đánh giá độ tương tự hành vi của Chương trình đạt mức tương đối so với hành vi thực tế.

Phép đo \textbf{SSE} là một phép đo cải tiến của phép đo \textbf{RS}, khi sử dụng kỹ thuật \textbf{DSE} dựa trên Chương trình tham chiếu để tạo các giá trị đầu vào thử nghiệm. Vì phải khám tất cả các nhánh đường đi của Chương trình tham chiếu nên tốc độ xử lý của phép đo \textbf{SSE} sẽ chậm và chiếm nhiều tài nguyên hơn phép đo \textbf{RS}. Tập dữ liệu đầu vào thử nghiệm được tạo bởi phép đo \textbf{SSE} có khả năng phủ tất cả các nhánh của Chương trình tham chiếu nên kết quả đánh giá độ tương tự hành vi của phép đo \textbf{SSE} sẽ chính xác hơn kết quả đánh giá độ tương tự hành vi của phép đo \textbf{RS}.

Phép đo \textbf{PSE} sử dụng kỹ thuật \textbf{DSE} khám phá tất cả các nhánh đường đi của Chương trình kết hợp để tạo ra tập dữ liệu đầu vào thử nghiệm chung cho cả hai chương trình. Vì vậy, phép đo \textbf{PSE} sẽ tốn nhiều thời gian thực thi chương trình và chiếm nhiều tài nguyên hơn phép đo \textbf{SSE}. Dữ liệu thử nghiệm của phép đo \textbf{PSE} sẽ có độ phủ cao hơn phép đo \textbf{SSE}, vì tất cả các giá trị đầu vào thử nghiệm có khả năng phủ tất cả các nhánh của Chương trình tham chiếu và Chương trình cần tình. Kết quả đánh giá độ tương tự hành vi của phép đo \textbf{PSE} sẽ chính xác hơn kết quả đánh giá độ tương tự của phép đo \textbf{SSE}.

\section*{Tổng kết chương}
Nội dung chính được trình bày trong chương này bao gồm những định nghĩa về thực thi chương trình, độ tương tự hành vi, sự khác biệt về hành vi độ tương tự về hành vi của chương trình. Mô tả, định nghĩa 3 kỹ thuật đo RS, SSE, PSE cũng như trình bày những ưu điểm, nhược điểm và hướng khắc phục của 3 kỹ thuật đo. Qua đó, giới thiệu một số tiêu chí đánh giá hiệu quả các kỹ thuật đo. 