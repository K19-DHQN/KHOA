\newpage
\chapter{GIỚI THIỆU}

\section{Lý do chọn đề tài, ngữ cảnh bài toán}
		
Trong những năm gần đây, xu hướng đào tạo lập trình viên nói riêng và công nghệ phần mềm nói chung đang ngày càng trở nên phổ biến. Các trường đại học và một số trung tâm đào tạo đã cho ra đời nhiều chương trình đào tạo phong phú về nội dung, đa dạng về hình thức và thu hút được nhiều sự quan tâm.
	
Trong mỗi khóa học, số lượng học viên thường có hằng trăm, hằng ngàn người tham gia, nhưng chỉ một vài giáo viên giảng dạy. Trong đó, chất lượng việc quản lý, truyền đạt nội dung của giáo viên và mức độ hiểu biết, nắm bắt nội dung chương trình học của học viên là yêu cầu tất cả các khóa học cần đạt được. Để đạt được điều đó, một yêu cầu bắt buộc đó là giáo viên phải đọc và hiểu tất cả các đoạn code của học sinh, nhưng công việc này lại tốn quá nhiều thời gian. Nếu bỏ qua hoặc trì hoãn các công việc như vậy thì người dạy không thể theo dõi được quá trình học tập của người học. Về phía người học, yêu cầu đặt ra là phải tiến bộ theo thời gian, nắm vững lý thuyết và thành thạo kỹ năng lập trình. Những không phải lúc nào gặp khó khăn người học đều có sự hỗ trợ kịp thời từ giảng viên. Họ có thể nhờ sự giúp đỡ từ đồng nghiệp, bạn bè, nhưng những người được nhờ hiups đỡ chưa chắc đã đủ trình độ, kinh nghiệm hoặc thời gian để ngồi bên cạnh giúp đỡ người học khi cần.
	
Để giảm bớt những khó khăn nêu trên, một công cụ hỗ trợ quá trình giảng dạy và học tập của giáo viên và học sinh hiệu quả hơn, tiết kiệm thời gian hơn đó là một công cụ tự động hóa (có thể một phần) việc đánh giá kết quả lập trình của học sinh, cũng như hỗ trợ theo giõi sự tiến bộ của học sinh. Công cụ tự động hóa này sẽ tính toán, định lượng tỷ lệ chính xác sự tương tự về hành vi giữa chương trình của người học và chương trình của người dạy đưa ra trước đó. Dựa trên kết quả, giáo viên sẽ đánh giá được kỹ năng lập trình của học sinh, sự giống nhau giữa hai chương trình càng cao thì tỷ lệ độ tương tự càng cao, điểm số càng cao. Nếu điểm số thấp, người học có thể quay lại kiểm tra để viết mã chương trình đúng hơn, hạn chế được nguy cơ tìm ẩn trong cách viết chương trình của người học.
	
Những đánh giá này có thể thực hiện được nếu ta đo được độ tương tự giữa các chương trình có độ chính xác cao. Đề tài “Độ tương tự về hành vi của các chương trình và làm thực nghiệm”  với mục đích sẽ giải quyết các vấn đề nếu trên.

\section{Những nghiên cứu có liên quan}


%%% Local Variables:
%%% mode: latex
%%% TeX-master: "Main"
%%% End:
