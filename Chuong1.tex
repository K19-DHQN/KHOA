\newpage
\chapter{GIỚI THIỆU}
\section{Giới thiệu chung}
Hiện nay, ngành Công nghệ thông tin với các chương trình đào tạo lập trình Online, kỹ sư phần mềm... đang trở nên rất phổ biến. Trên thế giới cũng có nhiều chương trình đào tạo nổi tiếng như Massive Open Online Courses (MOOC), edX, Coursera, Udacity thu hút hằng ngàn sinh viên theo học. Một số chương trình học lập trình online như Pex4Fun hay Code Hunt là một nền tảng học lập trình online thông qua trò chơi.\\

Những lớp học như thế này thường đặt ra một số thách thức, làm thế nào để kiểm soát được chất lượng học tập của người học. Trong khi các lớp học có hằng trăm người tham gia, nhưng thành viên tham giảng dạy chỉ vài người trong một lớp học. Hằng ngày, người giáo viên phải thường xuyên kiểm tra, nhắc nhở và phải đọc, nghiên cứu giúp đỡ cho học viên. Chỉ riêng việc đọc và hiểu code do học viên viết ra đã tốn quá nhiều thời gian của người dạy, nếu như bỏ qua thì khó có thể đánh giá được chất lượng của người học.\\

Để giảm bớt những vất vả, khó khăn của người dạy và người học. Một công cụ hỗ trợ, tự động đánh giá hành vi chương trình của người học sẽ giúp tiết kiệm được thời gian, giúp cho giáo viên việc quản lý chất lượng học tập của học viên được tốt hơn. Ví dụ, công cụ sẽ tự động đánh giá hành vi, so sánh hành vi chương trình của người học viết với hành vi chương trình của giáo viên. Nếu hành vi của hai chương trình có tỷ lệ giống nhau càng cao thì điểm số cho chương trình của người học càng cao. Ngược lại, nếu điểm số thấp người dạy và học sẽ tìm hiểu nguyên nhân vì sao và có hướng khác phục những hạn chế mà người học đang gặp phải. Đây cũng là một cách giúp cho người học không đi lệch khỏi kiến thức của chương trình đào tạo, tiết kiệm được thời gian cả người dạy và người học.\\

Ý tưởng trong việc đánh giá độ tương tự hành vi của hai chương trình đó là tính toán tìm ra một miền giá trị đầu vào chung cho cả hai chương trình. Đưa từng giá trị đầu vào chạy đồng thời trên cả hai chương trình và so sánh kết quả đầu ra của hai chương trình. Tuy nhiên, việc này sẽ không đạt, không khả thi, vì trong những trường hợp chương trình có miền giá trị đầu vào lớn hoặc vô hạn. Vì vậy, để đánh giá độ tương tự hành vi của hai chương trình khả thi hơn nếu chúng ta chỉ sử dụng một sô giá trị đầu vào đại diện cho toàn bộ miền giá trị đầu vào chung của hai chương trình.\\

Kỹ thuật Dynamic Symbolic Execution (DSE) là một kỹ thuật thu thập các ràng buộc từ các nhánh của chương trình, phủ nhận lại có hệ thống một phần các ràng buộc để tạo ra giá trị đầu vào cho một chương trình. Hiện nay, đã có nhiều công cụ sử dụng kỹ thuật DSE để giải quyết các ràng buộc một cách mạnh mẽ, tạo ra các giá trị đầu vào tin cậy có độ phủ cao.

\begin{center}
\begin{tabular}  {|c|c|c|} 
	\hline 
	\textbf{Tên Công cụ} & \textbf{Ngôn ngữ} & \textbf{Url} \\ 
	\hline 
	KLEE & LLVM & klee.github.io/ \\ 
	\hline 
	JPF	 & Java	& babelfish.arc.nasa.gov/trac/jpf \\
	\hline 
	jCUTE &	Java &	github.com/osl/jcute \\
	\hline 
	janala2	 & Java &	github.com/ksen007/janala2 \\
	\hline 
	JBSE	& Java	 & github.com/pietrobraione/jbse \\
	\hline 
	KeY &	Java &	www.key-project.org/ \\	
	\hline 
	Mayhem & 	Binary &	forallsecure.com/mayhem.html \\
	\hline 
	Otter &	C	& bitbucket.org/khooyp/otter/overview \\
	\hline 
	Rubyx & 	Ruby &	www.cs.umd.edu/~avik/papers/ssarorwa.pdf \\
	\hline 
	Pex	& .NET Framework	 & research.microsoft.com/en-us/projects/pex/ \\
	\hline 
	Jalangi2 &	JavaScript &	github.com/Samsung/jalangi2 \\
	\hline 
	Kite &	LLVM &	www.cs.ubc.ca/labs/isd/Projects/Kite/ \\
	\hline 
	pysymemu &	x86-64 / Native	 &github.com/feliam/pysymemu/ \\
	\hline 
	Triton	& x86 and x86-64 &	triton.quarkslab.com \\	
	\hline 
	BE-PUM &	x86	 & https://github.com/NMHai/BE-PUM	 \\	
	\hline

\end{tabular} 
\end{center}


	